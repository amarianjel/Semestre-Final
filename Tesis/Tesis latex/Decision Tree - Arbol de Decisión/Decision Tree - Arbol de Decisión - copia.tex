    \hypertarget{decision-tree---entrenamiento-del-algoritmo}{%
\section{Decision Tree - Entrenamiento del
algoritmo}\label{decision-tree---entrenamiento-del-algoritmo}}

Este algoritmo es utilizado pdecisiones te permite evaluar resultados,
costos y consecuencias de una decisión compleja en grandes volúmenes
datos y solventar problemas (Aplicación de un árbol de decisión difusa
con clasificación de ambigüedad para determinar el exceso de peso en
escolares).

En el entrenamiento del algoritmo el programa de Machine Learning
adquiere la información que trabajamos en los métodos anteriores. Es
aquí donde se obtendrá el conocimiento para futuras decisiones, es
importante asegurarse que las decisiones que sean tomadas posteriormente
al proceso de entrenamiento se añadan a la base de conocimiento del
algoritmo para futuras ejecuciones de este.

La BDD trabajada actualmente cuenta con:

    \begin{tcolorbox}[breakable, size=fbox, boxrule=1pt, pad at break*=1mm,colback=cellbackground, colframe=cellborder]
\prompt{In}{incolor}{53}{\boxspacing}
\begin{Verbatim}[commandchars=\\\{\}]
\PY{n+nb}{print}\PY{p}{(}\PY{l+s+s1}{\PYZsq{}}\PY{l+s+s1}{Existen }\PY{l+s+si}{\PYZob{}\PYZcb{}}\PY{l+s+s1}{ pacientes con }\PY{l+s+si}{\PYZob{}\PYZcb{}}\PY{l+s+s1}{ variables.}\PY{l+s+s1}{\PYZsq{}}\PY{o}{.}\PY{n}{format}\PY{p}{(}\PY{o}{*}\PY{n}{dataset}\PY{o}{.}\PY{n}{shape}\PY{p}{)}\PY{p}{)}
\PY{n+nb}{print}\PY{p}{(}\PY{l+s+s2}{\PYZdq{}}\PY{l+s+s2}{Existen}\PY{l+s+s2}{\PYZdq{}}\PY{p}{,} \PY{n}{dataset}\PY{o}{.}\PY{n}{size}\PY{p}{,} \PY{l+s+s2}{\PYZdq{}}\PY{l+s+s2}{elementos}\PY{l+s+s2}{\PYZdq{}}\PY{p}{)}
\end{Verbatim}
\end{tcolorbox}

    \begin{Verbatim}[commandchars=\\\{\}]
Existen 46 pacientes con 26 variables.
Existen 1196 elementos
    \end{Verbatim}

    \hypertarget{variable-categuxf3rica}{%
\subsection{Variable categórica}\label{variable-categuxf3rica}}

En el paso de la preparación de los datos de entrada, propusimos la
variable ``NIHSS alta ACV'' que podía poseer 42 valores diferentes, la
cual se clasificó y se transformó en ``NIHSS\_alta\_cat'' que contenía 6
categorías las que fueron reducidas a 1 variable con dos estados. Asi el
paciente tendrá un buen pronóstico o no con el nombre de la variable
``NIHSS\_alta\_ESTABLE\_O\_GRAVE''. Los estudios de Machine Learning no
sugieren tener variables binarias para nuestro estudio.

Las variables dependientes representan el rendimiento o conclusión que
se está estudiando. Las variables independientes, además conocidas en
una relación estadística como regresores, representan insumos o causas,
donde se encuentran las razones potenciales de alteración.

    \begin{tcolorbox}[breakable, size=fbox, boxrule=1pt, pad at break*=1mm,colback=cellbackground, colframe=cellborder]
\prompt{In}{incolor}{54}{\boxspacing}
\begin{Verbatim}[commandchars=\\\{\}]
\PY{c+c1}{\PYZsh{} variables objetivo e independientes:}
\PY{k+kn}{from} \PY{n+nn}{sklearn}\PY{n+nn}{.}\PY{n+nn}{model\PYZus{}selection} \PY{k+kn}{import} \PY{n}{train\PYZus{}test\PYZus{}split}

\PY{c+c1}{\PYZsh{} X son nuestras variables independientes}
\PY{n}{X} \PY{o}{=} \PY{n}{dataset}\PY{o}{.}\PY{n}{drop}\PY{p}{(}\PY{l+s+s1}{\PYZsq{}}\PY{l+s+s1}{NIHSS\PYZus{}alta\PYZus{}ESTABLE\PYZus{}O\PYZus{}GRAVE}\PY{l+s+s1}{\PYZsq{}}\PY{p}{,} \PY{n}{axis} \PY{o}{=} \PY{l+m+mi}{1}\PY{p}{)}

\PY{c+c1}{\PYZsh{} y es nuestra variable dependiente}
\PY{n}{y} \PY{o}{=} \PY{n}{dataset}\PY{p}{[}\PY{l+s+s1}{\PYZsq{}}\PY{l+s+s1}{NIHSS\PYZus{}alta\PYZus{}ESTABLE\PYZus{}O\PYZus{}GRAVE}\PY{l+s+s1}{\PYZsq{}}\PY{p}{]}

\PY{c+c1}{\PYZsh{} Uso de Skicit\PYZhy{}learn para dividir datos en conjuntos de entrenamiento y prueba }
\PY{c+c1}{\PYZsh{} División 75\PYZpc{} de datos para entrenamiento, 25\PYZpc{} de datos para test}
\PY{n}{X\PYZus{}train}\PY{p}{,} \PY{n}{X\PYZus{}test}\PY{p}{,} \PY{n}{y\PYZus{}train}\PY{p}{,} \PY{n}{y\PYZus{}test} \PY{o}{=} \PY{n}{train\PYZus{}test\PYZus{}split}\PY{p}{(}\PY{n}{X}\PY{p}{,} \PY{n}{y}\PY{p}{,} \PY{n}{random\PYZus{}state}\PY{o}{=}\PY{l+m+mi}{0}\PY{p}{)}
\end{Verbatim}
\end{tcolorbox}

    \hypertarget{creaciuxf3n-del-modelo-y-entrenamiento}{%
\subsection{Creación del modelo y
entrenamiento}\label{creaciuxf3n-del-modelo-y-entrenamiento}}

Para la creación del modelo se utilizará el modelo en la forma más
estándar posible, siendo que los modelos, antes del entrenamiento,
pueden recibir ajustes para manejar los datos de entrada, de una forma u
otra. Para que sea lo más parejo posible entre modelos se dejará de
forma estándar.

    \begin{tcolorbox}[breakable, size=fbox, boxrule=1pt, pad at break*=1mm,colback=cellbackground, colframe=cellborder]
\prompt{In}{incolor}{55}{\boxspacing}
\begin{Verbatim}[commandchars=\\\{\}]
\PY{k+kn}{from} \PY{n+nn}{sklearn}\PY{n+nn}{.}\PY{n+nn}{tree} \PY{k+kn}{import} \PY{n}{DecisionTreeClassifier}

\PY{c+c1}{\PYZsh{} Creamos el modelo de Arbol de Decisión (y configuramos el número máximo de nodos\PYZhy{}hoja)}
\PY{n}{dtc} \PY{o}{=} \PY{n}{DecisionTreeClassifier}\PY{p}{(}\PY{n}{criterion} \PY{o}{=} \PY{l+s+s1}{\PYZsq{}}\PY{l+s+s1}{gini}\PY{l+s+s1}{\PYZsq{}}\PY{p}{,} \PY{n}{random\PYZus{}state}\PY{o}{=}\PY{l+m+mi}{0}\PY{p}{)}
\PY{n}{dtc}\PY{o}{.}\PY{n}{fit}\PY{p}{(}\PY{n}{X\PYZus{}train}\PY{p}{,} \PY{n}{y\PYZus{}train}\PY{p}{)}
\end{Verbatim}
\end{tcolorbox}

            \begin{tcolorbox}[breakable, size=fbox, boxrule=.5pt, pad at break*=1mm, opacityfill=0]
\prompt{Out}{outcolor}{55}{\boxspacing}
\begin{Verbatim}[commandchars=\\\{\}]
DecisionTreeClassifier(random\_state=0)
\end{Verbatim}
\end{tcolorbox}
        
    \hypertarget{estructura-del-uxe1rbol-creado}{%
\subsubsection{Estructura del árbol
creado}\label{estructura-del-uxe1rbol-creado}}

    \begin{tcolorbox}[breakable, size=fbox, boxrule=1pt, pad at break*=1mm,colback=cellbackground, colframe=cellborder]
\prompt{In}{incolor}{56}{\boxspacing}
\begin{Verbatim}[commandchars=\\\{\}]
\PY{c+c1}{\PYZsh{} Estructura del árbol creado}
\PY{k+kn}{from} \PY{n+nn}{sklearn}\PY{n+nn}{.}\PY{n+nn}{tree} \PY{k+kn}{import} \PY{n}{plot\PYZus{}tree}

\PY{n}{fig}\PY{p}{,} \PY{n}{ax} \PY{o}{=} \PY{n}{plt}\PY{o}{.}\PY{n}{subplots}\PY{p}{(}\PY{n}{figsize}\PY{o}{=}\PY{p}{(}\PY{l+m+mi}{13}\PY{p}{,} \PY{l+m+mi}{6}\PY{p}{)}\PY{p}{)}

\PY{n+nb}{print}\PY{p}{(}\PY{l+s+sa}{f}\PY{l+s+s2}{\PYZdq{}}\PY{l+s+s2}{Profundidad del árbol: }\PY{l+s+si}{\PYZob{}}\PY{n}{dtc}\PY{o}{.}\PY{n}{get\PYZus{}depth}\PY{p}{(}\PY{p}{)}\PY{l+s+si}{\PYZcb{}}\PY{l+s+s2}{\PYZdq{}}\PY{p}{)}
\PY{n+nb}{print}\PY{p}{(}\PY{l+s+sa}{f}\PY{l+s+s2}{\PYZdq{}}\PY{l+s+s2}{Número de nodos terminales: }\PY{l+s+si}{\PYZob{}}\PY{n}{dtc}\PY{o}{.}\PY{n}{get\PYZus{}n\PYZus{}leaves}\PY{p}{(}\PY{p}{)}\PY{l+s+si}{\PYZcb{}}\PY{l+s+s2}{\PYZdq{}}\PY{p}{)}

\PY{n}{plot} \PY{o}{=} \PY{n}{plot\PYZus{}tree}\PY{p}{(}
            \PY{n}{decision\PYZus{}tree} \PY{o}{=} \PY{n}{dtc}\PY{p}{,}
            \PY{n}{feature\PYZus{}names} \PY{o}{=} \PY{n}{feature\PYZus{}list}\PY{p}{,}
            \PY{n}{class\PYZus{}names}   \PY{o}{=} \PY{l+s+s1}{\PYZsq{}}\PY{l+s+s1}{NIHSS\PYZus{}alta\PYZus{}ESTABLE\PYZus{}O\PYZus{}GRAVE}\PY{l+s+s1}{\PYZsq{}}\PY{p}{,}
            \PY{n}{filled}        \PY{o}{=} \PY{k+kc}{True}\PY{p}{,}
            \PY{n}{impurity}      \PY{o}{=} \PY{k+kc}{False}\PY{p}{,}
            \PY{n}{fontsize}      \PY{o}{=} \PY{l+m+mi}{10}\PY{p}{,}
            \PY{n}{ax}            \PY{o}{=} \PY{n}{ax}
       \PY{p}{)}
\end{Verbatim}
\end{tcolorbox}

    \begin{Verbatim}[commandchars=\\\{\}]
Profundidad del árbol: 3
Número de nodos terminales: 5
    \end{Verbatim}

\begin{center}
    	\begin{figure}[htb]
	\centering
    \adjustimage{max size={0.9\linewidth}{0.9\paperheight}}{Decision Tree - Arbol de Decisión/output_88_1.png}
	\caption{Árbol de decisión}
	\label{fig:ad}
	\end{figure}
\end{center}
    
    \hypertarget{predicciones-sobre-los-datos-de-prueba-y-muxe9tricas-de-rendimiento}{%
\subsection{Predicciones sobre los datos de prueba y métricas de
rendimiento}\label{predicciones-sobre-los-datos-de-prueba-y-muxe9tricas-de-rendimiento}}

Para llevar una forma más ordenada, es necesario crear las variables de
predicciones, para asi sacar las métricas de rendimiento más fácilmente.
Las métricas de rendimiento nos ofrecerán información de cómo se
comportó el algoritmo durante el entrenamiento, dando a conocer valores
importantes como lo son la precisión, exhaustividad, valor-F.

    \begin{tcolorbox}[breakable, size=fbox, boxrule=1pt, pad at break*=1mm,colback=cellbackground, colframe=cellborder]
\prompt{In}{incolor}{57}{\boxspacing}
\begin{Verbatim}[commandchars=\\\{\}]
\PY{c+c1}{\PYZsh{} Predicción Entrenamiento }
\PY{n}{prediccionEntreno} \PY{o}{=} \PY{n}{dtc}\PY{o}{.}\PY{n}{predict}\PY{p}{(}\PY{n}{X\PYZus{}train}\PY{p}{)}

\PY{c+c1}{\PYZsh{} Predicción Tests}
\PY{n}{prediccionTests} \PY{o}{=} \PY{n}{dtc}\PY{o}{.}\PY{n}{predict}\PY{p}{(}\PY{n}{X\PYZus{}test}\PY{p}{)}
\end{Verbatim}
\end{tcolorbox}

    \begin{tcolorbox}[breakable, size=fbox, boxrule=1pt, pad at break*=1mm,colback=cellbackground, colframe=cellborder]
\prompt{In}{incolor}{58}{\boxspacing}
\begin{Verbatim}[commandchars=\\\{\}]
\PY{k+kn}{from} \PY{n+nn}{sklearn} \PY{k+kn}{import} \PY{n}{metrics}

\PY{n+nb}{print}\PY{p}{(}\PY{l+s+s2}{\PYZdq{}}\PY{l+s+s2}{Entrenamiento \PYZhy{} Presición :}\PY{l+s+s2}{\PYZdq{}}\PY{p}{,} \PY{n}{metrics}\PY{o}{.}\PY{n}{accuracy\PYZus{}score}\PY{p}{(}\PY{n}{y\PYZus{}train}\PY{p}{,} \PY{n}{prediccionEntreno}\PY{p}{)}\PY{p}{)}
\PY{n+nb}{print}\PY{p}{(}\PY{l+s+s2}{\PYZdq{}}\PY{l+s+s2}{Entrenamiento \PYZhy{} Reporte de clasificación:}\PY{l+s+se}{\PYZbs{}n}\PY{l+s+s2}{\PYZdq{}}\PY{p}{,} \PY{n}{metrics}\PY{o}{.}\PY{n}{classification\PYZus{}report}\PY{p}{(}\PY{n}{y\PYZus{}train}\PY{p}{,} \PY{n}{prediccionEntreno}\PY{p}{)}\PY{p}{)}
\end{Verbatim}
\end{tcolorbox}

    \begin{Verbatim}[commandchars=\\\{\}]
Entrenamiento - Presición : 1.0
Entrenamiento - Reporte de clasificación:
               precision    recall  f1-score   support

           0       1.00      1.00      1.00        20
           1       1.00      1.00      1.00        14

    accuracy                           1.00        34
   macro avg       1.00      1.00      1.00        34
weighted avg       1.00      1.00      1.00        34

    \end{Verbatim}

    La precisión de los datos de entrenamiento en el modelo tiene un valor
excelente de 100\% de predicción, la exhaustividad informa la cantidad
de datos capaz de identificar y en este caso es de un 100\% de los datos
y finalmente el F1 combina los valores de precisión y exhaustividad
obteniéndose un 100\% igual. Todos los valores mencionados aplican para
los estados de la variable predictora.

    \hypertarget{matriz-de-confusiuxf3n}{%
\subsection{Matriz de Confusión}\label{matriz-de-confusiuxf3n}}

En el campo de la inteligencia artificial y en especial en el problema
de la clasificación estadística, una matriz de confusión es una
herramienta que permite la visualización del desempeño de un algoritmo
que se emplea en aprendizaje supervisado.

    \begin{tcolorbox}[breakable, size=fbox, boxrule=1pt, pad at break*=1mm,colback=cellbackground, colframe=cellborder]
\prompt{In}{incolor}{59}{\boxspacing}
\begin{Verbatim}[commandchars=\\\{\}]
\PY{k+kn}{from} \PY{n+nn}{matplotlib} \PY{k+kn}{import} \PY{n}{pyplot} \PY{k}{as} \PY{n}{plot}
\PY{k+kn}{from} \PY{n+nn}{mlxtend}\PY{n+nn}{.}\PY{n+nn}{plotting} \PY{k+kn}{import} \PY{n}{plot\PYZus{}confusion\PYZus{}matrix}
\PY{k+kn}{from} \PY{n+nn}{sklearn}\PY{n+nn}{.}\PY{n+nn}{metrics} \PY{k+kn}{import} \PY{n}{confusion\PYZus{}matrix}

\PY{n}{matriz} \PY{o}{=} \PY{n}{confusion\PYZus{}matrix}\PY{p}{(}\PY{n}{y\PYZus{}train}\PY{p}{,} \PY{n}{prediccionEntreno}\PY{p}{)}

\PY{n}{plot\PYZus{}confusion\PYZus{}matrix}\PY{p}{(}\PY{n}{conf\PYZus{}mat}\PY{o}{=}\PY{n}{matriz}\PY{p}{,} \PY{n}{figsize}\PY{o}{=}\PY{p}{(}\PY{l+m+mi}{6}\PY{p}{,}\PY{l+m+mi}{6}\PY{p}{)}\PY{p}{,} \PY{n}{show\PYZus{}normed}\PY{o}{=}\PY{k+kc}{False}\PY{p}{)}
\PY{n}{plot}\PY{o}{.}\PY{n}{tight\PYZus{}layout}\PY{p}{(}\PY{p}{)}
\end{Verbatim}
\end{tcolorbox}

\begin{center}
    	\begin{figure}[htb]
	\centering
    \adjustimage{max size={0.9\linewidth}{0.9\paperheight}}{Decision Tree - Arbol de Decisión/output_94_0.png}
	\caption{Matriz de confusión de entrenamiento Decision Tree}
	\label{fig:mcedt}
	\end{figure}
\end{center}
    
    En la matriz de confusión (1 , 1) podemos observar el resultado en el
que el modelo predice correctamente la clase positiva y en el (2, 2) el
resultado donde el modelo predice correctamente la clase negativa. Los
demás elementos de la matriz contiene valor nulo o 0, estos son los
errores de la predicción.

    \hypertarget{decision-tree---testeo-del-algoritmo}{%
\section{Decision Tree - Testeo del
algoritmo}\label{decision-tree---testeo-del-algoritmo}}

El testing tiene la finalidad de llevar a cabo la prueba si el modelo
funciona correctamente, identificando riesgos o erros que se produjeron
en los datos. No se realizará ajustes posteriores al testing para poder
comparar los algoritmos en la sección de resultados.

    \hypertarget{predicciones-sobre-los-datos-del-testing-y-muxe9tricas-de-rendimiento}{%
\subsection{Predicciones sobre los datos del testing y métricas de
rendimiento}\label{predicciones-sobre-los-datos-del-testing-y-muxe9tricas-de-rendimiento}}

Ahora es momento de evaluar los datos ya entrenados con el testing. Las
métricas de rendimiento nos ofrecerán información de cómo se comportó el
algoritmo durante el entrenamiento, dando a conocer valores importantes
como lo son la precisión, exhaustividad, valor-F.

    \begin{tcolorbox}[breakable, size=fbox, boxrule=1pt, pad at break*=1mm,colback=cellbackground, colframe=cellborder]
\prompt{In}{incolor}{60}{\boxspacing}
\begin{Verbatim}[commandchars=\\\{\}]
\PY{n+nb}{print}\PY{p}{(}\PY{l+s+s2}{\PYZdq{}}\PY{l+s+s2}{Tests \PYZhy{} Presición :}\PY{l+s+s2}{\PYZdq{}}\PY{p}{,} \PY{n}{metrics}\PY{o}{.}\PY{n}{accuracy\PYZus{}score}\PY{p}{(}\PY{n}{y\PYZus{}test}\PY{p}{,} \PY{n}{prediccionTests}\PY{p}{)}\PY{p}{)}
\PY{n+nb}{print}\PY{p}{(}\PY{l+s+s2}{\PYZdq{}}\PY{l+s+s2}{Tests \PYZhy{} Reporte de clasificación:}\PY{l+s+se}{\PYZbs{}n}\PY{l+s+s2}{\PYZdq{}}\PY{p}{,} \PY{n}{metrics}\PY{o}{.}\PY{n}{classification\PYZus{}report}\PY{p}{(}\PY{n}{y\PYZus{}test}\PY{p}{,} \PY{n}{prediccionTests}\PY{p}{)}\PY{p}{)}
\end{Verbatim}
\end{tcolorbox}

    \begin{Verbatim}[commandchars=\\\{\}]
Tests - Presición : 0.8333333333333334
Tests - Reporte de clasificación:
               precision    recall  f1-score   support

           0       0.83      0.83      0.83         6
           1       0.83      0.83      0.83         6

    accuracy                           0.83        12
   macro avg       0.83      0.83      0.83        12
weighted avg       0.83      0.83      0.83        12

    \end{Verbatim}

    La precisión de los datos del testing en el modelo tiene un valor de
83\% en ambos estados. La exhaustividad en el estado 0 alcanza un 83\%.
Por otra parte, el F1 combina los valores de precisión y exhaustividad
obteniéndose un 83\%.

Lo que se busca es la precisión del modelo, por consecuencia, el
Algoritmo de Machine Learning Decision Tree tiene una precisión del
83,3\% de predicción.

    \hypertarget{matriz-de-confusiuxf3n}{%
\subsection{Matriz de Confusión}\label{matriz-de-confusiuxf3n}}

Evaluaremos la matriz de confusión que se elaboró con los datos del
testing.

    \begin{tcolorbox}[breakable, size=fbox, boxrule=1pt, pad at break*=1mm,colback=cellbackground, colframe=cellborder]
\prompt{In}{incolor}{61}{\boxspacing}
\begin{Verbatim}[commandchars=\\\{\}]
\PY{n}{matriz} \PY{o}{=} \PY{n}{confusion\PYZus{}matrix}\PY{p}{(}\PY{n}{y\PYZus{}test}\PY{p}{,} \PY{n}{prediccionTests}\PY{p}{)}

\PY{n}{plot\PYZus{}confusion\PYZus{}matrix}\PY{p}{(}\PY{n}{conf\PYZus{}mat}\PY{o}{=}\PY{n}{matriz}\PY{p}{,} \PY{n}{figsize}\PY{o}{=}\PY{p}{(}\PY{l+m+mi}{6}\PY{p}{,}\PY{l+m+mi}{6}\PY{p}{)}\PY{p}{,} \PY{n}{show\PYZus{}normed}\PY{o}{=}\PY{k+kc}{False}\PY{p}{)}
\PY{n}{plt}\PY{o}{.}\PY{n}{tight\PYZus{}layout}\PY{p}{(}\PY{p}{)}
\end{Verbatim}
\end{tcolorbox}

\begin{center}
    	\begin{figure}[htb]
	\centering
    \adjustimage{max size={0.9\linewidth}{0.9\paperheight}}{Decision Tree - Arbol de Decisión/output_101_0.png}
	\caption{Matriz de confusión de testing Decision Tree}
	\label{fig:mctdt}
	\end{figure}
\end{center}
    
    En la matriz de confusión (1, 1) podemos observar el resultado en el que
el modelo predice correctamente la clase positiva con un alto valor y en
el (2, 2) el resultado donde el modelo predice correctamente la clase
negativa también con un alto valor, acercándose a la misma de la clase
positiva. Los demás elementos de la matriz contienen valor pequeño,
estos son los errores de la predicción.

Las afirmaciones anteriores sugieren que la las predicciones son altas y
el error es muy pequeño en la predicción.

    \hypertarget{decision-tree---uso-del-algoritmo}{%
\section{Decision Tree - Uso del
algoritmo}\label{decision-tree---uso-del-algoritmo}}

El último paso de la metodología es el uso del algoritmo, nosotros lo
utilizaremos para desarrollar probabilidades en los predictores. No
todos los modelos poseen los mismos métodos ni atributos, por ende, se
tratará de realizar comparaciones con métodos similares entre sí.

    \hypertarget{importancia-de-los-predictores}{%
\subsection{Importancia de los
predictores}\label{importancia-de-los-predictores}}

Por experiencia previa y contemplando los gráficos producidos en el paso
3, sabemos que algunas características no son útiles para nuestro
problema de predicción. Reducir la cantidad de funciones será la mejor
alternativa, lo que acotará el tiempo de ejecución, con suerte sin
comprometer significativamente el rendimiento, asi podemos examinar la
importancia de las funciones de nuestro modelo. La importancia de cada
predictor en el modelo se calcula como la reducción total (normalizada)
en el criterio de división. Si un predictor no ha sido seleccionado en
ninguna división, no se ha incluido en el modelo y por lo tanto su
importancia es 0.

    \begin{tcolorbox}[breakable, size=fbox, boxrule=1pt, pad at break*=1mm,colback=cellbackground, colframe=cellborder]
\prompt{In}{incolor}{62}{\boxspacing}
\begin{Verbatim}[commandchars=\\\{\}]
\PY{c+c1}{\PYZsh{} Predicciones probabilísticas}
\PY{c+c1}{\PYZsh{} =======================================}
\PY{c+c1}{\PYZsh{} Con .predict\PYZus{}proba() se obtiene, para cada observación, la probabilidad predicha}
\PY{c+c1}{\PYZsh{} de pertenecer a cada una de las dos clases.}
\PY{n}{predicciones} \PY{o}{=} \PY{n}{dtc}\PY{o}{.}\PY{n}{predict\PYZus{}proba}\PY{p}{(}\PY{n}{X\PYZus{}test}\PY{p}{)}
\PY{n}{predicciones} \PY{o}{=} \PY{n}{pd}\PY{o}{.}\PY{n}{DataFrame}\PY{p}{(}\PY{n}{predicciones}\PY{p}{,} \PY{n}{columns} \PY{o}{=} \PY{n}{dtc}\PY{o}{.}\PY{n}{classes\PYZus{}}\PY{p}{)}
\PY{n}{predicciones}
\end{Verbatim}
\end{tcolorbox}

            \begin{tcolorbox}[breakable, size=fbox, boxrule=.5pt, pad at break*=1mm, opacityfill=0]
\prompt{Out}{outcolor}{62}{\boxspacing}
\begin{Verbatim}[commandchars=\\\{\}]
      0    1
0   0.0  1.0
1   1.0  0.0
2   1.0  0.0
3   1.0  0.0
4   0.0  1.0
5   0.0  1.0
6   0.0  1.0
7   1.0  0.0
8   0.0  1.0
9   0.0  1.0
10  1.0  0.0
11  1.0  0.0
\end{Verbatim}
\end{tcolorbox}
        
    La predicción probabilistica en los estados 0 y 1 tuvo éxito al momento
de la desición. Los resultados arrojados son exactos.

    \begin{tcolorbox}[breakable, size=fbox, boxrule=1pt, pad at break*=1mm,colback=cellbackground, colframe=cellborder]
\prompt{In}{incolor}{63}{\boxspacing}
\begin{Verbatim}[commandchars=\\\{\}]
\PY{c+c1}{\PYZsh{} Predicciones con clasificación final}
\PY{c+c1}{\PYZsh{} =====================================}
\PY{c+c1}{\PYZsh{} Con .predict() se obtiene, para cada observación, la clasificación predicha por}
\PY{c+c1}{\PYZsh{} el modelo. Esta clasificación se corresponde con la clase con mayor probabilidad.}
\PY{n}{predicciones} \PY{o}{=} \PY{n}{dtc}\PY{o}{.}\PY{n}{predict}\PY{p}{(}\PY{n}{X\PYZus{}test}\PY{p}{)}
\PY{n}{predicciones} \PY{o}{=} \PY{n}{pd}\PY{o}{.}\PY{n}{DataFrame}\PY{p}{(}\PY{n}{predicciones}\PY{p}{)}
\PY{n}{predicciones}
\end{Verbatim}
\end{tcolorbox}

            \begin{tcolorbox}[breakable, size=fbox, boxrule=.5pt, pad at break*=1mm, opacityfill=0]
\prompt{Out}{outcolor}{63}{\boxspacing}
\begin{Verbatim}[commandchars=\\\{\}]
    0
0   1
1   0
2   0
3   0
4   1
5   1
6   1
7   0
8   1
9   1
10  0
11  0
\end{Verbatim}
\end{tcolorbox}
        
    Se observa un valor binario de 0 o 1, donde se muestra cada variable
desarrollada en el modelo puede tomar dicho valor. El valor 0 demuestra
que la tupla no logra predecir el estado 0 de la variable predictora, y
por el contrario, el estado 1 es que logra la predicción del estado en
esa tupla. Como reflejo de la tabla, éxiste un gran poder de predicción.

    \begin{tcolorbox}[breakable, size=fbox, boxrule=1pt, pad at break*=1mm,colback=cellbackground, colframe=cellborder]
\prompt{In}{incolor}{64}{\boxspacing}
\begin{Verbatim}[commandchars=\\\{\}]
\PY{n}{importancia\PYZus{}predictores} \PY{o}{=} \PY{n}{pd}\PY{o}{.}\PY{n}{DataFrame}\PY{p}{(}
                            \PY{p}{\PYZob{}}\PY{l+s+s1}{\PYZsq{}}\PY{l+s+s1}{Predictor}\PY{l+s+s1}{\PYZsq{}}\PY{p}{:} \PY{n}{dataset}\PY{o}{.}\PY{n}{drop}\PY{p}{(}\PY{n}{columns} \PY{o}{=} \PY{l+s+s2}{\PYZdq{}}\PY{l+s+s2}{NIHSS\PYZus{}alta\PYZus{}ESTABLE\PYZus{}O\PYZus{}GRAVE}\PY{l+s+s2}{\PYZdq{}}\PY{p}{)}\PY{o}{.}\PY{n}{columns}\PY{p}{,}
                             \PY{l+s+s1}{\PYZsq{}}\PY{l+s+s1}{importancia}\PY{l+s+s1}{\PYZsq{}}\PY{p}{:} \PY{n}{dtc}\PY{o}{.}\PY{n}{feature\PYZus{}importances\PYZus{}}\PY{p}{\PYZcb{}}
                            \PY{p}{)}
\PY{n+nb}{print}\PY{p}{(}\PY{l+s+s2}{\PYZdq{}}\PY{l+s+s2}{Importancia de los predictores en el modelo}\PY{l+s+s2}{\PYZdq{}}\PY{p}{)}
\PY{n+nb}{print}\PY{p}{(}\PY{l+s+s2}{\PYZdq{}}\PY{l+s+s2}{\PYZhy{}\PYZhy{}\PYZhy{}\PYZhy{}\PYZhy{}\PYZhy{}\PYZhy{}\PYZhy{}\PYZhy{}\PYZhy{}\PYZhy{}\PYZhy{}\PYZhy{}\PYZhy{}\PYZhy{}\PYZhy{}\PYZhy{}\PYZhy{}\PYZhy{}\PYZhy{}\PYZhy{}\PYZhy{}\PYZhy{}\PYZhy{}\PYZhy{}\PYZhy{}\PYZhy{}\PYZhy{}\PYZhy{}\PYZhy{}\PYZhy{}\PYZhy{}\PYZhy{}\PYZhy{}\PYZhy{}\PYZhy{}\PYZhy{}\PYZhy{}\PYZhy{}\PYZhy{}\PYZhy{}\PYZhy{}\PYZhy{}}\PY{l+s+s2}{\PYZdq{}}\PY{p}{)}
\PY{n}{importancia\PYZus{}predictores}\PY{o}{.}\PY{n}{sort\PYZus{}values}\PY{p}{(}\PY{l+s+s1}{\PYZsq{}}\PY{l+s+s1}{importancia}\PY{l+s+s1}{\PYZsq{}}\PY{p}{,} \PY{n}{ascending}\PY{o}{=}\PY{k+kc}{False}\PY{p}{)}
\end{Verbatim}
\end{tcolorbox}

    \begin{Verbatim}[commandchars=\\\{\}]
Importancia de los predictores en el modelo
-------------------------------------------
    \end{Verbatim}

            \begin{tcolorbox}[breakable, size=fbox, boxrule=.5pt, pad at break*=1mm, opacityfill=0]
\prompt{Out}{outcolor}{64}{\boxspacing}
\begin{Verbatim}[commandchars=\\\{\}]
                                      Predictor  importancia
11                               NIHSS\_alta\_cat     0.419841
10                               NIHSS alta ACV     0.215873
13                              CONTEO G.B.\_cat     0.208163
17                                  GLUCOSA\_cat     0.156122
0                                           HTA     0.000000
14                                      INR\_cat     0.000000
23  NIHSS\_INICIO\_cat\_Moderado (Buen Pronostico)     0.000000
22        NIHSS\_INICIO\_cat\_Leve (Trombolisando)     0.000000
21                       NIHSS\_INICIO\_cat\_Grave     0.000000
20              NIHSS\_INICIO\_cat\_Déficit Mínimo     0.000000
19          NIHSS\_INICIO\_cat\_Déficit Importante     0.000000
18                                     EDAD\_cat     0.000000
16                               COL. TOTAL\_cat     0.000000
15                            TRIGLICERIDOS\_cat     0.000000
12                                  GLASGOW\_cat     0.000000
1                                      DIABETES     0.000000
9                               NIHSS INICO ACV     0.000000
8                          GLASGOW AL INICO ACV     0.000000
7                                   CONTEO G.B.     0.000000
6                                           INR     0.000000
5                                 TRIGLICERIDOS     0.000000
4                                    COL. TOTAL     0.000000
3                                       GLUCOSA     0.000000
2                                          EDAD     0.000000
24                 NIHSS\_INICIO\_cat\_Sin Déficit     0.000000
\end{Verbatim}
\end{tcolorbox}
        
    \begin{tcolorbox}[breakable, size=fbox, boxrule=1pt, pad at break*=1mm,colback=cellbackground, colframe=cellborder]
\prompt{In}{incolor}{65}{\boxspacing}
\begin{Verbatim}[commandchars=\\\{\}]
\PY{c+c1}{\PYZsh{} Get numerical feature importances}
\PY{n}{importances} \PY{o}{=} \PY{n+nb}{list}\PY{p}{(}\PY{n}{dtc}\PY{o}{.}\PY{n}{feature\PYZus{}importances\PYZus{}}\PY{p}{)}
\PY{c+c1}{\PYZsh{} List of tuples with variable and importance}
\PY{n}{feature\PYZus{}importances} \PY{o}{=} \PY{p}{[}\PY{p}{(}\PY{n}{feature}\PY{p}{,} \PY{n+nb}{round}\PY{p}{(}\PY{n}{importance}\PY{p}{,} \PY{l+m+mi}{2}\PY{p}{)}\PY{p}{)} \PY{k}{for} \PY{n}{feature}\PY{p}{,} \PY{n}{importance} \PY{o+ow}{in} \PY{n+nb}{zip}\PY{p}{(}\PY{n}{feature\PYZus{}list}\PY{p}{,} \PY{n}{importances}\PY{p}{)}\PY{p}{]}
\PY{c+c1}{\PYZsh{} Sort the feature importances by most important first}
\PY{n}{feature\PYZus{}importances} \PY{o}{=} \PY{n+nb}{sorted}\PY{p}{(}\PY{n}{feature\PYZus{}importances}\PY{p}{,} \PY{n}{key} \PY{o}{=} \PY{k}{lambda} \PY{n}{x}\PY{p}{:} \PY{n}{x}\PY{p}{[}\PY{l+m+mi}{1}\PY{p}{]}\PY{p}{,} \PY{n}{reverse} \PY{o}{=} \PY{k+kc}{True}\PY{p}{)}

\PY{c+c1}{\PYZsh{} Reset style }
\PY{n}{plt}\PY{o}{.}\PY{n}{style}\PY{o}{.}\PY{n}{use}\PY{p}{(}\PY{l+s+s1}{\PYZsq{}}\PY{l+s+s1}{fivethirtyeight}\PY{l+s+s1}{\PYZsq{}}\PY{p}{)}

\PY{c+c1}{\PYZsh{} lista de x ubicaciones para trazar}
\PY{n}{x\PYZus{}values} \PY{o}{=} \PY{n+nb}{list}\PY{p}{(}\PY{n+nb}{range}\PY{p}{(}\PY{n+nb}{len}\PY{p}{(}\PY{n}{importances}\PY{p}{)}\PY{p}{)}\PY{p}{)}

\PY{c+c1}{\PYZsh{} Gráfico de barras}
\PY{n}{plt}\PY{o}{.}\PY{n}{bar}\PY{p}{(}\PY{n}{x\PYZus{}values}\PY{p}{,} \PY{n}{importances}\PY{p}{,} \PY{n}{orientation} \PY{o}{=} \PY{l+s+s1}{\PYZsq{}}\PY{l+s+s1}{vertical}\PY{l+s+s1}{\PYZsq{}}\PY{p}{,} \PY{n}{color} \PY{o}{=} \PY{l+s+s1}{\PYZsq{}}\PY{l+s+s1}{r}\PY{l+s+s1}{\PYZsq{}}\PY{p}{,} \PY{n}{edgecolor} \PY{o}{=} \PY{l+s+s1}{\PYZsq{}}\PY{l+s+s1}{k}\PY{l+s+s1}{\PYZsq{}}\PY{p}{,} \PY{n}{linewidth} \PY{o}{=} \PY{l+m+mf}{1.2}\PY{p}{)}

\PY{c+c1}{\PYZsh{} Marque las etiquetas para el eje x}
\PY{n}{plt}\PY{o}{.}\PY{n}{xticks}\PY{p}{(}\PY{n}{x\PYZus{}values}\PY{p}{,} \PY{n}{feature\PYZus{}list}\PY{p}{,} \PY{n}{rotation}\PY{o}{=}\PY{l+s+s1}{\PYZsq{}}\PY{l+s+s1}{vertical}\PY{l+s+s1}{\PYZsq{}}\PY{p}{)}

\PY{c+c1}{\PYZsh{} Etiquetas de eje y título}
\PY{n}{plt}\PY{o}{.}\PY{n}{ylabel}\PY{p}{(}\PY{l+s+s1}{\PYZsq{}}\PY{l+s+s1}{Importancia}\PY{l+s+s1}{\PYZsq{}}\PY{p}{)}\PY{p}{;} \PY{n}{plt}\PY{o}{.}\PY{n}{xlabel}\PY{p}{(}\PY{l+s+s1}{\PYZsq{}}\PY{l+s+s1}{Variable}\PY{l+s+s1}{\PYZsq{}}\PY{p}{)}\PY{p}{;} \PY{n}{plt}\PY{o}{.}\PY{n}{title}\PY{p}{(}\PY{l+s+s1}{\PYZsq{}}\PY{l+s+s1}{Importancia de Variables}\PY{l+s+s1}{\PYZsq{}}\PY{p}{)}\PY{p}{;}
\end{Verbatim}
\end{tcolorbox}

\begin{center}
    	\begin{figure}[htb]
	\centering

    \adjustimage{max size={0.9\linewidth}{0.9\paperheight}}{Decision Tree - Arbol de Decisión/output_110_0.png}
	\caption{Importancia de los predictores en Decision Tree}
	\label{fig:ipdt}
	\end{figure}
\end{center}
    
    Las variables que presentan gran importancia al momento de la predicción
son ``NIHSS\_alta\_cat'', ``NIHSS alta ACV'', ``CONTEO G.B.\_cat'' y
``GLUCOSA\_cat'' con más del 98\%, siendo que el mayor predictor esta
con un 41,9\% que es ``NIHSS\_alta\_cat. Se ve reflejado en el grafico
la gran importancia que representan las variables mencionadas para el
modelo.

    \hypertarget{importancia-acumulada}{%
\subsection{Importancia acumulada}\label{importancia-acumulada}}

Ahora reduciremos la cantidad de funciones en uso por el modelo a solo
aquellas requeridas para representar el 95\% de la importancia. Se debe
usar el mismo número de características en los conjuntos de
entrenamiento y prueba.

    \begin{tcolorbox}[breakable, size=fbox, boxrule=1pt, pad at break*=1mm,colback=cellbackground, colframe=cellborder]
\prompt{In}{incolor}{66}{\boxspacing}
\begin{Verbatim}[commandchars=\\\{\}]
\PY{c+c1}{\PYZsh{} Lista de funciones ordenadas de mayor a menor importancia}
\PY{n}{sorted\PYZus{}importances} \PY{o}{=} \PY{p}{[}\PY{n}{importance}\PY{p}{[}\PY{l+m+mi}{1}\PY{p}{]} \PY{k}{for} \PY{n}{importance} \PY{o+ow}{in} \PY{n}{feature\PYZus{}importances}\PY{p}{]}
\PY{n}{sorted\PYZus{}features} \PY{o}{=} \PY{p}{[}\PY{n}{importance}\PY{p}{[}\PY{l+m+mi}{0}\PY{p}{]} \PY{k}{for} \PY{n}{importance} \PY{o+ow}{in} \PY{n}{feature\PYZus{}importances}\PY{p}{]}

\PY{c+c1}{\PYZsh{} Importancias acumulativas}
\PY{n}{cumulative\PYZus{}importances} \PY{o}{=} \PY{n}{np}\PY{o}{.}\PY{n}{cumsum}\PY{p}{(}\PY{n}{sorted\PYZus{}importances}\PY{p}{)}

\PY{c+c1}{\PYZsh{} Haz un gráfico de líneas}
\PY{n}{plt}\PY{o}{.}\PY{n}{plot}\PY{p}{(}\PY{n}{x\PYZus{}values}\PY{p}{,} \PY{n}{cumulative\PYZus{}importances}\PY{p}{,} \PY{l+s+s1}{\PYZsq{}}\PY{l+s+s1}{g\PYZhy{}}\PY{l+s+s1}{\PYZsq{}}\PY{p}{)}

\PY{c+c1}{\PYZsh{} Dibujar línea al 95\PYZpc{} de importancia retenida}
\PY{n}{plt}\PY{o}{.}\PY{n}{hlines}\PY{p}{(}\PY{n}{y} \PY{o}{=} \PY{l+m+mf}{0.95}\PY{p}{,} \PY{n}{xmin}\PY{o}{=}\PY{l+m+mi}{0}\PY{p}{,} \PY{n}{xmax}\PY{o}{=}\PY{n+nb}{len}\PY{p}{(}\PY{n}{sorted\PYZus{}importances}\PY{p}{)}\PY{p}{,} \PY{n}{color} \PY{o}{=} \PY{l+s+s1}{\PYZsq{}}\PY{l+s+s1}{r}\PY{l+s+s1}{\PYZsq{}}\PY{p}{,} \PY{n}{linestyles} \PY{o}{=} \PY{l+s+s1}{\PYZsq{}}\PY{l+s+s1}{dashed}\PY{l+s+s1}{\PYZsq{}}\PY{p}{)}

\PY{c+c1}{\PYZsh{} Formato x ticks y etiquetas}
\PY{n}{plt}\PY{o}{.}\PY{n}{xticks}\PY{p}{(}\PY{n}{x\PYZus{}values}\PY{p}{,} \PY{n}{sorted\PYZus{}features}\PY{p}{,} \PY{n}{rotation} \PY{o}{=} \PY{l+s+s1}{\PYZsq{}}\PY{l+s+s1}{vertical}\PY{l+s+s1}{\PYZsq{}}\PY{p}{)}

\PY{c+c1}{\PYZsh{} Etiquetas de eje y título}
\PY{n}{plt}\PY{o}{.}\PY{n}{xlabel}\PY{p}{(}\PY{l+s+s1}{\PYZsq{}}\PY{l+s+s1}{Variable}\PY{l+s+s1}{\PYZsq{}}\PY{p}{)}\PY{p}{;} \PY{n}{plt}\PY{o}{.}\PY{n}{ylabel}\PY{p}{(}\PY{l+s+s1}{\PYZsq{}}\PY{l+s+s1}{Importancia acumulada}\PY{l+s+s1}{\PYZsq{}}\PY{p}{)}\PY{p}{;} \PY{n}{plt}\PY{o}{.}\PY{n}{title}\PY{p}{(}\PY{l+s+s1}{\PYZsq{}}\PY{l+s+s1}{Importancia acumulada}\PY{l+s+s1}{\PYZsq{}}\PY{p}{)}\PY{p}{;}
\end{Verbatim}
\end{tcolorbox}

\begin{center}
    	\begin{figure}[htb]
	\centering
    \adjustimage{max size={0.9\linewidth}{0.9\paperheight}}{Decision Tree - Arbol de Decisión/output_113_0.png}
	\caption{Importancia de los predictores acumulada en Decision Tree}
	\label{fig:ipadt}
	\end{figure}
\end{center}
    
    \begin{tcolorbox}[breakable, size=fbox, boxrule=1pt, pad at break*=1mm,colback=cellbackground, colframe=cellborder]
\prompt{In}{incolor}{67}{\boxspacing}
\begin{Verbatim}[commandchars=\\\{\}]
\PY{c+c1}{\PYZsh{} Encuentre el número de características para una importancia acumulada del 95\PYZpc{}}
\PY{c+c1}{\PYZsh{} Agregue 1 porque Python está indexado a cero}
\PY{n+nb}{print}\PY{p}{(}\PY{l+s+s1}{\PYZsq{}}\PY{l+s+s1}{Número de columna para el 95 }\PY{l+s+si}{\PYZpc{} d}\PY{l+s+s1}{e importancia:}\PY{l+s+s1}{\PYZsq{}}\PY{p}{,} \PY{n}{np}\PY{o}{.}\PY{n}{where}\PY{p}{(}\PY{n}{cumulative\PYZus{}importances} \PY{o}{\PYZgt{}} \PY{l+m+mf}{0.95}\PY{p}{)}\PY{p}{[}\PY{l+m+mi}{0}\PY{p}{]}\PY{p}{[}\PY{l+m+mi}{0}\PY{p}{]} \PY{o}{+} \PY{l+m+mi}{1}\PY{p}{)}
\end{Verbatim}
\end{tcolorbox}

    \begin{Verbatim}[commandchars=\\\{\}]
Número de columna para el 95 \% de importancia: 4
    \end{Verbatim}

    En el gráfico de importancia acumulada la curva se dispara con la
``GLUCOSA\_cat'' y se ratifica con el resultado del 95\% de importancia
medido anteriormente.


    % Add a bibliography block to the postdoc


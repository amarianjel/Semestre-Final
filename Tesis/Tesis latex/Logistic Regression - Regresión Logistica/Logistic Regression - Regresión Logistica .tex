    \hypertarget{logistic-regression---entrenamiento-del-algoritmo}{%
\section{Logistic Regression - Entrenamiento del
algoritmo}\label{logistic-regression---entrenamiento-del-algoritmo}}

Este algoritmo de clasificación utilizado para predecir la probabilidad
de una variable dependiente categórica. En la regresión logística, la
variable dependiente es una variable binaria que contiene datos
codificados como 1-0, sí-no, abierto-cerrado, etc(Libro del metodo).

En el entrenamiento del algoritmo el programa de Machine Learning
adquiere la información que trabajamos en los métodos anteriores. Es
aquí donde se obtendrá el conocimiento para futuras decisiones, es
importante asegurarse que las decisiones que sean tomadas posteriormente
al proceso de entrenamiento se añadan a la base de conocimiento del
algoritmo para futuras ejecuciones de este.

La BDD trabajada actualmente cuenta con:

    \begin{tcolorbox}[breakable, size=fbox, boxrule=1pt, pad at break*=1mm,colback=cellbackground, colframe=cellborder]
\prompt{In}{incolor}{54}{\boxspacing}
\begin{Verbatim}[commandchars=\\\{\}]
\PY{n+nb}{print}\PY{p}{(}\PY{l+s+s1}{\PYZsq{}}\PY{l+s+s1}{Existen }\PY{l+s+si}{\PYZob{}\PYZcb{}}\PY{l+s+s1}{ pacientes con }\PY{l+s+si}{\PYZob{}\PYZcb{}}\PY{l+s+s1}{ variables.}\PY{l+s+s1}{\PYZsq{}}\PY{o}{.}\PY{n}{format}\PY{p}{(}\PY{o}{*}\PY{n}{dataset}\PY{o}{.}\PY{n}{shape}\PY{p}{)}\PY{p}{)}
\PY{n+nb}{print}\PY{p}{(}\PY{l+s+s2}{\PYZdq{}}\PY{l+s+s2}{Existen}\PY{l+s+s2}{\PYZdq{}}\PY{p}{,} \PY{n}{dataset}\PY{o}{.}\PY{n}{size}\PY{p}{,} \PY{l+s+s2}{\PYZdq{}}\PY{l+s+s2}{elementos}\PY{l+s+s2}{\PYZdq{}}\PY{p}{)}
\end{Verbatim}
\end{tcolorbox}

    \begin{Verbatim}[commandchars=\\\{\}]
Existen 46 pacientes con 26 variables.
Existen 1196 elementos
    \end{Verbatim}

    \hypertarget{variable-categuxf3rica}{%
\subsection{Variable categórica}\label{variable-categuxf3rica}}

En el paso de la preparación de los datos de entrada, propusimos la
variable ``NIHSS alta ACV'' que podía poseer 42 valores diferentes, la
cual se clasificó y se transformó en ``NIHSS\_alta\_cat'' que contenía 6
categorías las que fueron reducidas a 1 variable con dos estados. Asi el
paciente tendrá un buen pronóstico o no con el nombre de la variable
``NIHSS\_alta\_ESTABLE\_O\_GRAVE''. Los estudios de Machine Learning no
sugieren tener variables binarias para nuestro estudio.

Las variables dependientes representan el rendimiento o conclusión que
se está estudiando. Las variables independientes, además conocidas en
una relación estadística como regresores, representan insumos o causas,
donde se encuentran las razones potenciales de alteración.

    \begin{tcolorbox}[breakable, size=fbox, boxrule=1pt, pad at break*=1mm,colback=cellbackground, colframe=cellborder]
\prompt{In}{incolor}{55}{\boxspacing}
\begin{Verbatim}[commandchars=\\\{\}]
\PY{c+c1}{\PYZsh{} variables objetivo e independientes:}
\PY{k+kn}{from} \PY{n+nn}{sklearn}\PY{n+nn}{.}\PY{n+nn}{model\PYZus{}selection} \PY{k+kn}{import} \PY{n}{train\PYZus{}test\PYZus{}split}

\PY{c+c1}{\PYZsh{} X son nuestras variables independientes}
\PY{n}{X} \PY{o}{=} \PY{n}{dataset}\PY{o}{.}\PY{n}{drop}\PY{p}{(}\PY{l+s+s1}{\PYZsq{}}\PY{l+s+s1}{NIHSS\PYZus{}alta\PYZus{}ESTABLE\PYZus{}O\PYZus{}GRAVE}\PY{l+s+s1}{\PYZsq{}}\PY{p}{,} \PY{n}{axis} \PY{o}{=} \PY{l+m+mi}{1}\PY{p}{)}

\PY{c+c1}{\PYZsh{} y es nuestra variable dependiente}
\PY{n}{y} \PY{o}{=} \PY{n}{dataset}\PY{p}{[}\PY{l+s+s1}{\PYZsq{}}\PY{l+s+s1}{NIHSS\PYZus{}alta\PYZus{}ESTABLE\PYZus{}O\PYZus{}GRAVE}\PY{l+s+s1}{\PYZsq{}}\PY{p}{]}

\PY{c+c1}{\PYZsh{} Uso de Skicit\PYZhy{}learn para dividir datos en conjuntos de entrenamiento y prueba }
\PY{c+c1}{\PYZsh{} División 75\PYZpc{} de datos para entrenamiento, 25\PYZpc{} de datos para test}
\PY{n}{X\PYZus{}train}\PY{p}{,} \PY{n}{X\PYZus{}test}\PY{p}{,} \PY{n}{y\PYZus{}train}\PY{p}{,} \PY{n}{y\PYZus{}test} \PY{o}{=} \PY{n}{train\PYZus{}test\PYZus{}split}\PY{p}{(}\PY{n}{X}\PY{p}{,} \PY{n}{y}\PY{p}{,} \PY{n}{test\PYZus{}size}\PY{o}{=}\PY{l+m+mf}{0.8}\PY{p}{,} \PY{n}{random\PYZus{}state}\PY{o}{=}\PY{l+m+mi}{0}\PY{p}{)}
\end{Verbatim}
\end{tcolorbox}

    \hypertarget{creaciuxf3n-del-modelo-y-entrenamiento}{%
\subsection{Creación del modelo y
entrenamiento}\label{creaciuxf3n-del-modelo-y-entrenamiento}}

Para la creación del modelo se utilizará el modelo en la forma más
estándar posible, siendo que los modelos, antes del entrenamiento,
pueden recibir ajustes para manejar los datos de entrada, de una forma u
otra. Para que sea lo más parejo posible entre modelos se dejará de
forma estándar.

    \begin{tcolorbox}[breakable, size=fbox, boxrule=1pt, pad at break*=1mm,colback=cellbackground, colframe=cellborder]
\prompt{In}{incolor}{56}{\boxspacing}
\begin{Verbatim}[commandchars=\\\{\}]
\PY{k+kn}{from} \PY{n+nn}{sklearn}\PY{n+nn}{.}\PY{n+nn}{linear\PYZus{}model} \PY{k+kn}{import} \PY{n}{LogisticRegression}
\PY{k+kn}{from} \PY{n+nn}{sklearn}\PY{n+nn}{.}\PY{n+nn}{model\PYZus{}selection} \PY{k+kn}{import} \PY{n}{train\PYZus{}test\PYZus{}split} \PY{c+c1}{\PYZsh{}separa las metricas}
\PY{k+kn}{from} \PY{n+nn}{sklearn} \PY{k+kn}{import} \PY{n}{metrics}

\PY{c+c1}{\PYZsh{} Creamos el modelo}
\PY{n}{lr} \PY{o}{=} \PY{n}{LogisticRegression}\PY{p}{(}\PY{n}{solver}\PY{o}{=}\PY{l+s+s1}{\PYZsq{}}\PY{l+s+s1}{lbfgs}\PY{l+s+s1}{\PYZsq{}}\PY{p}{,} \PY{n}{random\PYZus{}state}\PY{o}{=}\PY{l+m+mi}{0}\PY{p}{)}
\PY{n}{lr}\PY{o}{.}\PY{n}{fit}\PY{p}{(}\PY{n}{X\PYZus{}train}\PY{p}{,} \PY{n}{y\PYZus{}train}\PY{p}{)}
\end{Verbatim}
\end{tcolorbox}

            \begin{tcolorbox}[breakable, size=fbox, boxrule=.5pt, pad at break*=1mm, opacityfill=0]
\prompt{Out}{outcolor}{56}{\boxspacing}
\begin{Verbatim}[commandchars=\\\{\}]
LogisticRegression(random\_state=0)
\end{Verbatim}
\end{tcolorbox}
        
    \hypertarget{predicciones-sobre-los-datos-de-prueba-y-muxe9tricas-de-rendimiento}{%
\subsection{Predicciones sobre los datos de prueba y métricas de
rendimiento}\label{predicciones-sobre-los-datos-de-prueba-y-muxe9tricas-de-rendimiento}}

Para llevar una forma más ordenada, es necesario crear las variables de
predicciones, para asi sacar las métricas de rendimiento más fácilmente.
Las métricas de rendimiento nos ofrecerán información de cómo se
comportó el algoritmo durante el entrenamiento, dando a conocer valores
importantes como lo son la precisión, exhaustividad, valor-F.

    \begin{tcolorbox}[breakable, size=fbox, boxrule=1pt, pad at break*=1mm,colback=cellbackground, colframe=cellborder]
\prompt{In}{incolor}{57}{\boxspacing}
\begin{Verbatim}[commandchars=\\\{\}]
\PY{c+c1}{\PYZsh{} Predicción Entrenamiento }
\PY{n}{prediccionEntreno} \PY{o}{=} \PY{n}{lr}\PY{o}{.}\PY{n}{predict}\PY{p}{(}\PY{n}{X\PYZus{}train}\PY{p}{)}

\PY{c+c1}{\PYZsh{} Predicción Tests}
\PY{n}{prediccionTests} \PY{o}{=} \PY{n}{lr}\PY{o}{.}\PY{n}{predict}\PY{p}{(}\PY{n}{X\PYZus{}test}\PY{p}{)}
\end{Verbatim}
\end{tcolorbox}

    \begin{tcolorbox}[breakable, size=fbox, boxrule=1pt, pad at break*=1mm,colback=cellbackground, colframe=cellborder]
\prompt{In}{incolor}{58}{\boxspacing}
\begin{Verbatim}[commandchars=\\\{\}]
\PY{k+kn}{from} \PY{n+nn}{sklearn} \PY{k+kn}{import} \PY{n}{metrics}

\PY{n+nb}{print}\PY{p}{(}\PY{l+s+s2}{\PYZdq{}}\PY{l+s+s2}{Entrenamiento \PYZhy{} Presición :}\PY{l+s+s2}{\PYZdq{}}\PY{p}{,} \PY{n}{metrics}\PY{o}{.}\PY{n}{accuracy\PYZus{}score}\PY{p}{(}\PY{n}{y\PYZus{}train}\PY{p}{,} \PY{n}{prediccionEntreno}\PY{p}{)}\PY{p}{)}
\PY{n+nb}{print}\PY{p}{(}\PY{l+s+s2}{\PYZdq{}}\PY{l+s+s2}{Entrenamiento \PYZhy{} Reporte de clasificación:}\PY{l+s+se}{\PYZbs{}n}\PY{l+s+s2}{\PYZdq{}}\PY{p}{,} \PY{n}{metrics}\PY{o}{.}\PY{n}{classification\PYZus{}report}\PY{p}{(}\PY{n}{y\PYZus{}train}\PY{p}{,} \PY{n}{prediccionEntreno}\PY{p}{)}\PY{p}{)}
\end{Verbatim}
\end{tcolorbox}

    \begin{Verbatim}[commandchars=\\\{\}]
Entrenamiento - Presición : 1.0
Entrenamiento - Reporte de clasificación:
               precision    recall  f1-score   support

           0       1.00      1.00      1.00         6
           1       1.00      1.00      1.00         3

    accuracy                           1.00         9
   macro avg       1.00      1.00      1.00         9
weighted avg       1.00      1.00      1.00         9

    \end{Verbatim}

    La precisión de los datos de entrenamiento en el modelo tiene un valor
excelente de 100\% de predicción, la exhaustividad informa la cantidad
de datos capaz de identificar y en este caso es de un 100\% de los datos
y finalmente el F1 combina los valores de precisión y exhaustividad
obteniéndose un 100\% igual. Todos los valores mencionados aplican para
los estados de la variable predictora.

    \hypertarget{matriz-de-confusiuxf3n}{%
\subsection{Matriz de Confusión}\label{matriz-de-confusiuxf3n}}

En el campo de la inteligencia artificial y en especial en el problema
de la clasificación estadística, una matriz de confusión es una
herramienta que permite la visualización del desempeño de un algoritmo
que se emplea en aprendizaje supervisado.

    \begin{tcolorbox}[breakable, size=fbox, boxrule=1pt, pad at break*=1mm,colback=cellbackground, colframe=cellborder]
\prompt{In}{incolor}{59}{\boxspacing}
\begin{Verbatim}[commandchars=\\\{\}]
\PY{k+kn}{from} \PY{n+nn}{matplotlib} \PY{k+kn}{import} \PY{n}{pyplot} \PY{k}{as} \PY{n}{plot}
\PY{k+kn}{from} \PY{n+nn}{mlxtend}\PY{n+nn}{.}\PY{n+nn}{plotting} \PY{k+kn}{import} \PY{n}{plot\PYZus{}confusion\PYZus{}matrix}
\PY{k+kn}{from} \PY{n+nn}{sklearn}\PY{n+nn}{.}\PY{n+nn}{metrics} \PY{k+kn}{import} \PY{n}{confusion\PYZus{}matrix}

\PY{n}{matriz} \PY{o}{=} \PY{n}{confusion\PYZus{}matrix}\PY{p}{(}\PY{n}{y\PYZus{}train}\PY{p}{,} \PY{n}{prediccionEntreno}\PY{p}{)}

\PY{n}{plot\PYZus{}confusion\PYZus{}matrix}\PY{p}{(}\PY{n}{conf\PYZus{}mat}\PY{o}{=}\PY{n}{matriz}\PY{p}{,} \PY{n}{figsize}\PY{o}{=}\PY{p}{(}\PY{l+m+mi}{6}\PY{p}{,}\PY{l+m+mi}{6}\PY{p}{)}\PY{p}{,} \PY{n}{show\PYZus{}normed}\PY{o}{=}\PY{k+kc}{False}\PY{p}{)}
\PY{n}{plot}\PY{o}{.}\PY{n}{tight\PYZus{}layout}\PY{p}{(}\PY{p}{)}
\end{Verbatim}
\end{tcolorbox}

\begin{center}
    	\begin{figure}[htb]
	\centering
    \adjustimage{max size={0.9\linewidth}{0.9\paperheight}}{Logistic Regression - Regresión Logistica/output_90_0.png}
	\caption{Matriz de confusión de entrenamiento Logistic Regression}
	\label{fig:mcelr}
	\end{figure}
\end{center}
    
    En la matriz de confusión (1, 1) podemos observar el resultado en el que
el modelo predice correctamente la clase positiva y en el (2, 2) el
resultado donde el modelo predice correctamente la clase negativa. Los
demás elementos de la matriz contienen valor nulo o 0, estos son los
errores de la predicción.

    \hypertarget{logistic-regression---testeo-del-algoritmo}{%
\section{Logistic Regression - Testeo del
algoritmo}\label{logistic-regression---testeo-del-algoritmo}}

El testing tiene la finalidad de llevar a cabo la prueba si el modelo
funciona correctamente, identificando riesgos o erros que se produjeron
en los datos. No se realizará ajustes posteriores al testing para poder
comparar los algoritmos en la sección de resultados.

    \hypertarget{predicciones-sobre-los-datos-del-testing-y-muxe9tricas-de-rendimiento}{%
\subsection{Predicciones sobre los datos del testing y métricas de
rendimiento}\label{predicciones-sobre-los-datos-del-testing-y-muxe9tricas-de-rendimiento}}

Ahora es momento de evaluar los datos ya entrenados con el testing. Las
métricas de rendimiento nos ofrecerán información de cómo se comportó el
algoritmo durante el entrenamiento, dando a conocer valores importantes
como lo son la precisión, exhaustividad, valor-F.

    \begin{tcolorbox}[breakable, size=fbox, boxrule=1pt, pad at break*=1mm,colback=cellbackground, colframe=cellborder]
\prompt{In}{incolor}{60}{\boxspacing}
\begin{Verbatim}[commandchars=\\\{\}]
\PY{n+nb}{print}\PY{p}{(}\PY{l+s+s2}{\PYZdq{}}\PY{l+s+s2}{Tests \PYZhy{} Presición :}\PY{l+s+s2}{\PYZdq{}}\PY{p}{,} \PY{n}{metrics}\PY{o}{.}\PY{n}{accuracy\PYZus{}score}\PY{p}{(}\PY{n}{y\PYZus{}test}\PY{p}{,} \PY{n}{prediccionTests}\PY{p}{)}\PY{p}{)}
\PY{n+nb}{print}\PY{p}{(}\PY{l+s+s2}{\PYZdq{}}\PY{l+s+s2}{Tests \PYZhy{} Reporte de clasificación:}\PY{l+s+se}{\PYZbs{}n}\PY{l+s+s2}{\PYZdq{}}\PY{p}{,} \PY{n}{metrics}\PY{o}{.}\PY{n}{classification\PYZus{}report}\PY{p}{(}\PY{n}{y\PYZus{}test}\PY{p}{,} \PY{n}{prediccionTests}\PY{p}{)}\PY{p}{)}
\end{Verbatim}
\end{tcolorbox}

    \begin{Verbatim}[commandchars=\\\{\}]
Tests - Presición : 0.5945945945945946
Tests - Reporte de clasificación:
               precision    recall  f1-score   support

           0       0.59      0.80      0.68        20
           1       0.60      0.35      0.44        17

    accuracy                           0.59        37
   macro avg       0.60      0.58      0.56        37
weighted avg       0.60      0.59      0.57        37

    \end{Verbatim}

    La precisión de los datos del testing en el modelo tiene un valor de
59\% de predicción para el estado 0 y un 60\% de predicción para el
estado 1. La exhaustividad en el estado 0 alcanza el 80\% de los datos y
en el estado 1 alcanza solo el 35\%. Por otra parte, el F1 combina los
valores de precisión y exhaustividad obteniéndose un 68\% en el estado 0
y un 44\% en el estado 1.

Lo que se busca es la precisión del modelo, por consecuencia, el
Algoritmo de Machine Learning Logistic Regression tiene una precisión
del 59,4\% de predicción.

    \hypertarget{matriz-de-confusiuxf3n}{%
\subsection{Matriz de Confusión}\label{matriz-de-confusiuxf3n}}

Evaluaremos la matriz de confusión que se elaboró con los datos del
testing.

    \begin{tcolorbox}[breakable, size=fbox, boxrule=1pt, pad at break*=1mm,colback=cellbackground, colframe=cellborder]
\prompt{In}{incolor}{61}{\boxspacing}
\begin{Verbatim}[commandchars=\\\{\}]
\PY{n}{matriz} \PY{o}{=} \PY{n}{confusion\PYZus{}matrix}\PY{p}{(}\PY{n}{y\PYZus{}test}\PY{p}{,} \PY{n}{prediccionTests}\PY{p}{)}

\PY{n}{plot\PYZus{}confusion\PYZus{}matrix}\PY{p}{(}\PY{n}{conf\PYZus{}mat}\PY{o}{=}\PY{n}{matriz}\PY{p}{,} \PY{n}{figsize}\PY{o}{=}\PY{p}{(}\PY{l+m+mi}{6}\PY{p}{,}\PY{l+m+mi}{6}\PY{p}{)}\PY{p}{,} \PY{n}{show\PYZus{}normed}\PY{o}{=}\PY{k+kc}{False}\PY{p}{)}
\PY{n}{plt}\PY{o}{.}\PY{n}{tight\PYZus{}layout}\PY{p}{(}\PY{p}{)}
\end{Verbatim}
\end{tcolorbox}

\begin{center}
    	\begin{figure}[htb]
	\centering
    \adjustimage{max size={0.9\linewidth}{0.9\paperheight}}{Logistic Regression - Regresión Logistica/output_97_0.png}
	\caption{Matriz de confusión de testing Logistic Regression }
	\label{fig:aNISSh}
	\end{figure}
\end{center}
    
    En la matriz de confusión (1, 1) podemos observar el resultado en el que
el modelo predice correctamente la clase positiva con un alto valor y en
el (2, 2) el resultado donde el modelo predice correctamente la clase
negativa también con un alto valor, acercándose a la misma de la clase
positiva. los demás elementos de la matriz contienen valor pequeño,
estos son los errores de la predicción.

Las afirmaciones anteriores sugieren que la las predicciones son altas,
pero también existen errores en la predicción y el principal error recae
en el que el modelo predice erroneamnete la clase negativa cuando en
realidad es positiva.

    \hypertarget{logistic-regression---uso-del-algoritmo}{%
\section{Logistic Regression - Uso del
algoritmo}\label{logistic-regression---uso-del-algoritmo}}

El último paso de la metodología es el uso del algoritmo, nosotros lo
utilizaremos para desarrollar probabilidades en los predictores. No
todos los modelos poseen los mismos métodos ni atributos, por ende, se
tratará de realizar comparaciones con métodos similares entre sí.

    \hypertarget{importancia-de-los-predictores}{%
\subsection{Importancia de los
predictores}\label{importancia-de-los-predictores}}

Por experiencia previa y contemplando los gráficos producidos en el paso
3, sabemos que algunas características no son útiles para nuestro
problema de predicción. Reducir la cantidad de funciones será la mejor
alternativa, lo que acotará el tiempo de ejecución, con suerte sin
comprometer significativamente el rendimiento, asi podemos examinar la
importancia de las funciones de nuestro modelo. La importancia de cada
predictor en el modelo se calcula como la reducción total (normalizada)
en el criterio de división. Si un predictor no ha sido seleccionado en
ninguna división, no se ha incluido en el modelo y por lo tanto su
importancia es 0.

    \begin{tcolorbox}[breakable, size=fbox, boxrule=1pt, pad at break*=1mm,colback=cellbackground, colframe=cellborder]
\prompt{In}{incolor}{62}{\boxspacing}
\begin{Verbatim}[commandchars=\\\{\}]
\PY{c+c1}{\PYZsh{} Predicciones probabilísticas}
\PY{c+c1}{\PYZsh{} ==============================================================================}
\PY{c+c1}{\PYZsh{} Con .predict\PYZus{}proba() se obtiene, para cada observación, la probabilidad predicha}
\PY{c+c1}{\PYZsh{} de pertenecer a cada una de las dos clases.}
\PY{n}{predicciones} \PY{o}{=} \PY{n}{lr}\PY{o}{.}\PY{n}{predict\PYZus{}proba}\PY{p}{(}\PY{n}{X\PYZus{}test}\PY{p}{)}
\PY{n}{predicciones} \PY{o}{=} \PY{n}{pd}\PY{o}{.}\PY{n}{DataFrame}\PY{p}{(}\PY{n}{predicciones}\PY{p}{,} \PY{n}{columns} \PY{o}{=} \PY{n}{lr}\PY{o}{.}\PY{n}{classes\PYZus{}}\PY{p}{)}
\PY{n}{predicciones}
\end{Verbatim}
\end{tcolorbox}

            \begin{tcolorbox}[breakable, size=fbox, boxrule=.5pt, pad at break*=1mm, opacityfill=0]
\prompt{Out}{outcolor}{62}{\boxspacing}
\begin{Verbatim}[commandchars=\\\{\}]
           0             1
0   1.000000  9.893824e-09
1   1.000000  1.866854e-09
2   0.953234  4.676604e-02
3   0.999837  1.634762e-04
4   0.816220  1.837802e-01
5   0.999932  6.773520e-05
6   0.024484  9.755158e-01
7   0.998096  1.903587e-03
8   0.778886  2.211138e-01
9   0.000036  9.999642e-01
10  0.980607  1.939259e-02
11  0.996999  3.001220e-03
12  0.994224  5.775827e-03
13  0.999994  6.210286e-06
14  1.000000  1.525695e-15
15  0.142960  8.570396e-01
16  0.002332  9.976683e-01
17  0.001081  9.989187e-01
18  0.005275  9.947247e-01
19  0.999973  2.695853e-05
20  0.991790  8.210169e-03
21  0.955307  4.469293e-02
22  0.738349  2.616514e-01
23  0.998427  1.573032e-03
24  0.980207  1.979256e-02
25  1.000000  2.915765e-11
26  0.080143  9.198566e-01
27  0.000833  9.991667e-01
28  0.999996  4.440745e-06
29  0.949616  5.038360e-02
30  0.991810  8.190380e-03
31  0.990110  9.890212e-03
32  0.998938  1.062018e-03
33  1.000000  8.146734e-16
34  0.000053  9.999472e-01
35  0.999998  1.997562e-06
36  0.037317  9.626828e-01
\end{Verbatim}
\end{tcolorbox}
        
    Este método acepta un solo argumento que corresponde a los datos sobre
los cuales se calculan las probabilidades y devuelve una matriz de
listas que contienen las probabilidades de clase para los puntos de
datos de entrada. En este caso particular podemos observar que los
estados de la variable predictora tienen un valor de porcentaje
predictor, por ejemplo la tupla 0 posee un 100\% de precisión para el
estado 0 y un 0,98\% y fracción para el estado 1.

    \begin{tcolorbox}[breakable, size=fbox, boxrule=1pt, pad at break*=1mm,colback=cellbackground, colframe=cellborder]
\prompt{In}{incolor}{63}{\boxspacing}
\begin{Verbatim}[commandchars=\\\{\}]
\PY{c+c1}{\PYZsh{} Predicciones con clasificación final}
\PY{c+c1}{\PYZsh{} ==============================================================================}
\PY{c+c1}{\PYZsh{} Con .predict() se obtiene, para cada observación, la clasificación predicha por}
\PY{c+c1}{\PYZsh{} el modelo. Esta clasificación se corresponde con la clase con mayor probabilidad.}
\PY{n}{predicciones} \PY{o}{=} \PY{n}{lr}\PY{o}{.}\PY{n}{predict}\PY{p}{(}\PY{n}{X\PYZus{}test}\PY{p}{)}
\PY{n}{predicciones} \PY{o}{=} \PY{n}{pd}\PY{o}{.}\PY{n}{DataFrame}\PY{p}{(}\PY{n}{predicciones}\PY{p}{)}
\PY{n}{predicciones}
\end{Verbatim}
\end{tcolorbox}

            \begin{tcolorbox}[breakable, size=fbox, boxrule=.5pt, pad at break*=1mm, opacityfill=0]
\prompt{Out}{outcolor}{63}{\boxspacing}
\begin{Verbatim}[commandchars=\\\{\}]
    0
0   0
1   0
2   0
3   0
4   0
5   0
6   1
7   0
8   0
9   1
10  0
11  0
12  0
13  0
14  0
15  1
16  1
17  1
18  1
19  0
20  0
21  0
22  0
23  0
24  0
25  0
26  1
27  1
28  0
29  0
30  0
31  0
32  0
33  0
34  1
35  0
36  1
\end{Verbatim}
\end{tcolorbox}
        
    Se observa un valor binario de 0 o 1, donde se muestra cada variable
desarrollada en el modelo puede tomar dicho valor. El valor 0 demuestra
que la tupla no logra predecir el estado 0 de la variable predictora, y
por el contrario, el estado 1 es que logra la predicción del estado en
esa tupla.


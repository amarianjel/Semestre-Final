    \hypertarget{logistic-regression---testeo-del-algoritmo}{%
\section{Logistic Regression - Testeo del
algoritmo}\label{logistic-regression---testeo-del-algoritmo}}

El testing tiene la finalidad de llevar a cabo la prueba si el modelo
funciona correctamente, identificando riesgos o erros que se produjeron
en los datos. No se realizará ajustes posteriores al testing para poder
comparar los algoritmos en la sección de resultados.

    \hypertarget{predicciones-sobre-los-datos-del-testing-y-muxe9tricas-de-rendimiento}{%
\subsection{Predicciones sobre los datos del testing y métricas de
rendimiento}\label{predicciones-sobre-los-datos-del-testing-y-muxe9tricas-de-rendimiento}}

Ahora es momento de evaluar los datos ya entrenados con el testing. Las
métricas de rendimiento nos ofrecerán información de cómo se comportó el
algoritmo durante el entrenamiento, dando a conocer valores importantes
como lo son la precisión, exhaustividad, valor-F.

    \begin{tcolorbox}[breakable, size=fbox, boxrule=1pt, pad at break*=1mm,colback=cellbackground, colframe=cellborder]
\prompt{In}{incolor}{60}{\boxspacing}
\begin{Verbatim}[commandchars=\\\{\}]
\PY{n+nb}{print}\PY{p}{(}\PY{l+s+s2}{\PYZdq{}}\PY{l+s+s2}{Tests \PYZhy{} Presición :}\PY{l+s+s2}{\PYZdq{}}\PY{p}{,} \PY{n}{metrics}\PY{o}{.}\PY{n}{accuracy\PYZus{}score}\PY{p}{(}\PY{n}{y\PYZus{}test}\PY{p}{,} \PY{n}{prediccionTests}\PY{p}{)}\PY{p}{)}
\PY{n+nb}{print}\PY{p}{(}\PY{l+s+s2}{\PYZdq{}}\PY{l+s+s2}{Tests \PYZhy{} Reporte de clasificación:}\PY{l+s+se}{\PYZbs{}n}\PY{l+s+s2}{\PYZdq{}}\PY{p}{,} \PY{n}{metrics}\PY{o}{.}\PY{n}{classification\PYZus{}report}\PY{p}{(}\PY{n}{y\PYZus{}test}\PY{p}{,} \PY{n}{prediccionTests}\PY{p}{)}\PY{p}{)}
\end{Verbatim}
\end{tcolorbox}

    \begin{Verbatim}[commandchars=\\\{\}]
Tests - Presición : 0.5945945945945946
Tests - Reporte de clasificación:
               precision    recall  f1-score   support

           0       0.59      0.80      0.68        20
           1       0.60      0.35      0.44        17

    accuracy                           0.59        37
   macro avg       0.60      0.58      0.56        37
weighted avg       0.60      0.59      0.57        37

    \end{Verbatim}

    La precisión de los datos del testing en el modelo tiene un valor de
59\% de predicción para el estado 0 y un 60\% de predicción para el
estado 1. La exhaustividad en el estado 0 alcanza el 80\% de los datos y
en el estado 1 alcanza solo el 35\%. Por otra parte, el F1 combina los
valores de precisión y exhaustividad obteniéndose un 68\% en el estado 0
y un 44\% en el estado 1.

Lo que se busca es la precisión del modelo, por consecuencia, el
Algoritmo de ML Logistic Regression tiene una precisión
del 59,4\% de predicción.

    \hypertarget{matriz-de-confusiuxf3n}{%
\subsection{Matriz de Confusión}\label{matriz-de-confusiuxf3n}}

Evaluaremos la matriz de confusión que se elaboró con los datos del
testing.

    \begin{tcolorbox}[breakable, size=fbox, boxrule=1pt, pad at break*=1mm,colback=cellbackground, colframe=cellborder]
\prompt{In}{incolor}{61}{\boxspacing}
\begin{Verbatim}[commandchars=\\\{\}]
\PY{n}{matriz} \PY{o}{=} \PY{n}{confusion\PYZus{}matrix}\PY{p}{(}\PY{n}{y\PYZus{}test}\PY{p}{,} \PY{n}{prediccionTests}\PY{p}{)}

\PY{n}{plot\PYZus{}confusion\PYZus{}matrix}\PY{p}{(}\PY{n}{conf\PYZus{}mat}\PY{o}{=}\PY{n}{matriz}\PY{p}{,} \PY{n}{figsize}\PY{o}{=}\PY{p}{(}\PY{l+m+mi}{6}\PY{p}{,}\PY{l+m+mi}{6}\PY{p}{)}\PY{p}{,} \PY{n}{show\PYZus{}normed}\PY{o}{=}\PY{k+kc}{False}\PY{p}{)}
\PY{n}{plt}\PY{o}{.}\PY{n}{tight\PYZus{}layout}\PY{p}{(}\PY{p}{)}
\end{Verbatim}
\end{tcolorbox}

\begin{center}
    	\begin{figure}[H]
	\centering
    \adjustimage{max size={0.5\linewidth}{0.9\paperheight}}{Logistic Regression - Regresión Logistica/output_97_0.png}
	\caption{Matriz de confusión de testing Logistic Regression }
	\label{fig:aNISSh}
	\end{figure}
\end{center}
    
    En la matriz de confusión \ref{fig:aNISSh} (1, 1) podemos observar el resultado en el que
el modelo predice correctamente la clase positiva con un alto valor y en
el (2, 2) el resultado donde el modelo predice correctamente la clase
negativa también con un alto valor, acercándose a la misma de la clase
positiva. los demás elementos de la matriz contienen valor pequeño,
estos son los errores de la predicción.

Las afirmaciones anteriores sugieren que la las predicciones son altas,
pero también existen errores en la predicción y el principal error recae
en el que el modelo predice erroneamnete la clase negativa cuando en
realidad es positiva.

    \hypertarget{logistic-regression---uso-del-algoritmo}{%
\section{Logistic Regression - Uso del
algoritmo}\label{logistic-regression---uso-del-algoritmo}}

El último paso de la metodología es el uso del algoritmo, nosotros lo
utilizaremos para desarrollar probabilidades en los predictores. No
todos los modelos poseen los mismos métodos ni atributos, por ende, se
tratará de realizar comparaciones con métodos similares entre sí.

    \hypertarget{importancia-de-los-predictores}{%
\subsection{Importancia de los
predictores}\label{importancia-de-los-predictores}}

Por experiencia previa y contemplando los gráficos producidos en el paso
3, sabemos que algunas características no son útiles para nuestro
problema de predicción. Reducir la cantidad de funciones será la mejor
alternativa, lo que acotará el tiempo de ejecución, con suerte sin
comprometer significativamente el rendimiento, asi podemos examinar la
importancia de las funciones de nuestro modelo. La importancia de cada
predictor en el modelo se calcula como la reducción total (normalizada)
en el criterio de división. Si un predictor no ha sido seleccionado en
ninguna división, no se ha incluido en el modelo y por lo tanto su
importancia es 0.

    \begin{tcolorbox}[breakable, size=fbox, boxrule=1pt, pad at break*=1mm,colback=cellbackground, colframe=cellborder]
\prompt{In}{incolor}{62}{\boxspacing}
\begin{Verbatim}[commandchars=\\\{\}]
\PY{c+c1}{\PYZsh{} Predicciones probabilísticas}
\PY{c+c1}{\PYZsh{} ==============================================================================}
\PY{c+c1}{\PYZsh{} Con .predict\PYZus{}proba() se obtiene, para cada observación, la probabilidad predicha}
\PY{c+c1}{\PYZsh{} de pertenecer a cada una de las dos clases.}
\PY{n}{predicciones} \PY{o}{=} \PY{n}{lr}\PY{o}{.}\PY{n}{predict\PYZus{}proba}\PY{p}{(}\PY{n}{X\PYZus{}test}\PY{p}{)}
\PY{n}{predicciones} \PY{o}{=} \PY{n}{pd}\PY{o}{.}\PY{n}{DataFrame}\PY{p}{(}\PY{n}{predicciones}\PY{p}{,} \PY{n}{columns} \PY{o}{=} \PY{n}{lr}\PY{o}{.}\PY{n}{classes\PYZus{}}\PY{p}{)}
\PY{n}{predicciones}
\end{Verbatim}
\end{tcolorbox}

\begin{table}[H]
\centering
\setlength{\tabcolsep}{5pt}
\resizebox{0.3\textwidth}{!}{
\begin{tabular}{|c|l|l|}
\hline
\multicolumn{1}{|l|}{} & \multicolumn{1}{c|}{\textbf{0}} & \multicolumn{1}{c|}{\textbf{1}} \\ \hline \hline
\textbf{0} & 1 & 9,89E-09 \\ \hline
\textbf{1} & 1 & 1,87E-09 \\ \hline
\textbf{2} & 0,953234 & 0,046766 \\ \hline
\textbf{3} & 0,999837 & 0,000163 \\ \hline
\textbf{4} & 0,81622 & 0,18378 \\ \hline
\textbf{5} & 0,999932 & 6,77E-05 \\ \hline
\textbf{6} & 0,024484 & 0,975516 \\ \hline
\textbf{7} & 0,998096 & 0,001904 \\ \hline
\textbf{8} & 0,778886 & 0,221114 \\ \hline
\textbf{9} & 3,58E-05 & 0,999964 \\ \hline
\textbf{10} & 0,980607 & 0,019393 \\ \hline
\textbf{11} & 0,996999 & 0,003001 \\ \hline
\textbf{12} & 0,994224 & 0,005776 \\ \hline
\textbf{13} & 0,999994 & 6,21E-06 \\ \hline
\textbf{14} & 1 & 1,53E-15 \\ \hline
\textbf{15} & 0,14296 & 0,85704 \\ \hline
\textbf{16} & 0,002332 & 0,997668 \\ \hline
\textbf{17} & 0,001081 & 0,998919 \\ \hline
\textbf{18} & 0,005275 & 0,994725 \\ \hline
\textbf{19} & 0,999973 & 2,7E-05 \\ \hline
\textbf{20} & 0,99179 & 0,00821 \\ \hline
\textbf{21} & 0,955307 & 0,044693 \\ \hline
\textbf{22} & 0,738349 & 0,261651 \\ \hline
\textbf{23} & 0,998427 & 0,001573 \\ \hline
\textbf{24} & 0,980207 & 0,019793 \\ \hline
\textbf{25} & 1 & 2,92E-11 \\ \hline
\textbf{26} & 0,080143 & 0,919857 \\ \hline
\textbf{27} & 0,000833 & 0,999167 \\ \hline
\textbf{28} & 0,999996 & 4,44E-06 \\ \hline
\textbf{29} & 0,949616 & 0,050384 \\ \hline
\textbf{30} & 0,99181 & 0,00819 \\ \hline
\textbf{31} & 0,99011 & 0,00989 \\ \hline
\textbf{32} & 0,998938 & 0,001062 \\ \hline
\textbf{33} & 1 & 8,15E-16 \\ \hline
\textbf{34} & 5,28E-05 & 0,999947 \\ \hline
\textbf{35} & 0,999998 & 2E-06 \\ \hline
\textbf{36} & 0,037317 & 0,962683 \\ \hline
\end{tabular}%
}
\caption{Predicciones probabilísticas para cada observación Logistic Regression}
\label{tab:clasificacin lr}
\end{table}
        
    Este método acepta un solo argumento que corresponde a los datos sobre
los cuales se calculan las probabilidades y devuelve una matriz de
listas que contienen las probabilidades de clase para los puntos de
datos de entrada. En este caso la tabla \ref{tab:clasificacin lr} podemos observar que los
estados de la variable predictora tienen un valor de porcentaje
predictor, por ejemplo la tupla 0 posee un 100\% de precisión para el
estado 0 y un 0,98\% y fracción para el estado 1.

    \begin{tcolorbox}[breakable, size=fbox, boxrule=1pt, pad at break*=1mm,colback=cellbackground, colframe=cellborder]
\prompt{In}{incolor}{63}{\boxspacing}
\begin{Verbatim}[commandchars=\\\{\}]
\PY{c+c1}{\PYZsh{} Predicciones con clasificación final}
\PY{c+c1}{\PYZsh{} ==============================================================================}
\PY{c+c1}{\PYZsh{} Con .predict() se obtiene, para cada observación, la clasificación predicha por}
\PY{c+c1}{\PYZsh{} el modelo. Esta clasificación se corresponde con la clase con mayor probabilidad.}
\PY{n}{predicciones} \PY{o}{=} \PY{n}{lr}\PY{o}{.}\PY{n}{predict}\PY{p}{(}\PY{n}{X\PYZus{}test}\PY{p}{)}
\PY{n}{predicciones} \PY{o}{=} \PY{n}{pd}\PY{o}{.}\PY{n}{DataFrame}\PY{p}{(}\PY{n}{predicciones}\PY{p}{)}
\PY{n}{predicciones}
\end{Verbatim}
\end{tcolorbox}

\begin{table}[H]
\centering
\setlength{\tabcolsep}{10pt}
\resizebox{0.15\textwidth}{!}{
\begin{tabular}{|c|l|}
\hline
\multicolumn{1}{|l|}{} & \multicolumn{1}{c|}{\textbf{0}} \\ \hline \hline
\textbf{0} & 0 \\ \hline
\textbf{1} & 0 \\ \hline
\textbf{2} & 0 \\ \hline
\textbf{3} & 0 \\ \hline
\textbf{4} & 0 \\ \hline
\textbf{5} & 0 \\ \hline
\textbf{6} & 1 \\ \hline
\textbf{7} & 0 \\ \hline
\textbf{8} & 0 \\ \hline
\textbf{9} & 1 \\ \hline
\textbf{10} & 0 \\ \hline
\textbf{11} & 0 \\ \hline
\textbf{12} & 0 \\ \hline
\textbf{13} & 0 \\ \hline
\textbf{14} & 0 \\ \hline
\textbf{15} & 1 \\ \hline
\textbf{16} & 1 \\ \hline
\textbf{17} & 1 \\ \hline
\textbf{18} & 1 \\ \hline
\textbf{19} & 0 \\ \hline
\textbf{20} & 0 \\ \hline
\textbf{21} & 0 \\ \hline
\textbf{22} & 0 \\ \hline
\textbf{23} & 0 \\ \hline
\textbf{24} & 0 \\ \hline
\textbf{25} & 0 \\ \hline
\textbf{26} & 1 \\ \hline
\textbf{27} & 1 \\ \hline
\textbf{28} & 0 \\ \hline
\textbf{29} & 0 \\ \hline
\textbf{30} & 0 \\ \hline
\textbf{31} & 0 \\ \hline
\textbf{32} & 0 \\ \hline
\textbf{33} & 0 \\ \hline
\textbf{34} & 1 \\ \hline
\textbf{35} & 0 \\ \hline
\textbf{36} & 1 \\ \hline
\end{tabular}%
}
\caption{Predicciones probabilísticas con clasificación final Logistic Regression}
\label{tab:clasificacion lr}
\end{table}
        
    Se observa en la tabla \ref{tab:clasificacion lr} un valor binario de 0 o 1, donde se muestra cada variable
desarrollada en el modelo puede tomar dicho valor. El valor 0 demuestra
que la tupla no logra predecir el estado 0 de la variable predictora, y
por el contrario, el estado 1 es que logra la predicción del estado en
esa tupla.


    \hypertarget{nauxefve-bayes---entrenamiento-del-algoritmo}{%
\section{Naïve Bayes - Entrenamiento del
algoritmo}\label{nauxefve-bayes---entrenamiento-del-algoritmo}}

Este algoritmo es utilizado para predecir grandes volúmenes de datos. En
este caso no se cuenta con un gran volumen de datos. El clasificador
Naïve-Bayes aprende de los datos de entrenamiento y luego predice la
clase de la instancia de prueba con la mayor probabilidad posterior.
También es útil para datos dimensionales altos ya que la probabilidad de
cada atributo se estima independientemente (Maquinas de soporte
vectorial y Naïve bayes y algoritmos geneticops).

En el entrenamiento del algoritmo el programa de Machine Learning
adquiere la información que trabajamos en los métodos anteriores. Es
aquí donde se obtendrá el conocimiento para futuras decisiones, es
importante asegurarse que las decisiones que sean tomadas posteriormente
al proceso de entrenamiento se añadan a la base de conocimiento del
algoritmo para futuras ejecuciones de este.

La BDD trabajada actualmente cuenta con:

    \begin{tcolorbox}[breakable, size=fbox, boxrule=1pt, pad at break*=1mm,colback=cellbackground, colframe=cellborder]
\prompt{In}{incolor}{53}{\boxspacing}
\begin{Verbatim}[commandchars=\\\{\}]
\PY{n+nb}{print}\PY{p}{(}\PY{l+s+s1}{\PYZsq{}}\PY{l+s+s1}{Existen }\PY{l+s+si}{\PYZob{}\PYZcb{}}\PY{l+s+s1}{ pacientes con }\PY{l+s+si}{\PYZob{}\PYZcb{}}\PY{l+s+s1}{ variables.}\PY{l+s+s1}{\PYZsq{}}\PY{o}{.}\PY{n}{format}\PY{p}{(}\PY{o}{*}\PY{n}{dataset}\PY{o}{.}\PY{n}{shape}\PY{p}{)}\PY{p}{)}
\PY{n+nb}{print}\PY{p}{(}\PY{l+s+s2}{\PYZdq{}}\PY{l+s+s2}{Existen}\PY{l+s+s2}{\PYZdq{}}\PY{p}{,} \PY{n}{dataset}\PY{o}{.}\PY{n}{size}\PY{p}{,} \PY{l+s+s2}{\PYZdq{}}\PY{l+s+s2}{elementos}\PY{l+s+s2}{\PYZdq{}}\PY{p}{)}
\end{Verbatim}
\end{tcolorbox}

    \begin{Verbatim}[commandchars=\\\{\}]
Existen 46 pacientes con 26 variables.
Existen 1196 elementos
    \end{Verbatim}

    \hypertarget{variable-categuxf3rica}{%
\subsection{Variable categórica}\label{variable-categuxf3rica}}

En el paso de la preparación de los datos de entrada, propusimos la
variable ``NIHSS alta ACV'' que podía poseer 42 valores diferentes, la
cual se clasificó y se transformó en ``NIHSS\_alta\_cat'' que contenía 6
categorías las que fueron reducidas a 1 variable con dos estados. Asi el
paciente tendrá un buen pronóstico o no con el nombre de la variable
``NIHSS\_alta\_ESTABLE\_O\_GRAVE''. Los estudios de Machine Learning no
sugieren tener variables binarias para nuestro estudio.

Las variables dependientes representan el rendimiento o conclusión que
se está estudiando. Las variables independientes, además conocidas en
una relación estadística como regresores, representan insumos o causas,
donde se encuentran las razones potenciales de alteración.

    \begin{tcolorbox}[breakable, size=fbox, boxrule=1pt, pad at break*=1mm,colback=cellbackground, colframe=cellborder]
\prompt{In}{incolor}{54}{\boxspacing}
\begin{Verbatim}[commandchars=\\\{\}]
\PY{c+c1}{\PYZsh{} variables objetivo e independientes:}
\PY{k+kn}{from} \PY{n+nn}{sklearn}\PY{n+nn}{.}\PY{n+nn}{model\PYZus{}selection} \PY{k+kn}{import} \PY{n}{train\PYZus{}test\PYZus{}split}

\PY{c+c1}{\PYZsh{} X son nuestras variables independientes}
\PY{n}{X} \PY{o}{=} \PY{n}{dataset}\PY{o}{.}\PY{n}{drop}\PY{p}{(}\PY{l+s+s1}{\PYZsq{}}\PY{l+s+s1}{NIHSS\PYZus{}alta\PYZus{}ESTABLE\PYZus{}O\PYZus{}GRAVE}\PY{l+s+s1}{\PYZsq{}}\PY{p}{,} \PY{n}{axis} \PY{o}{=} \PY{l+m+mi}{1}\PY{p}{)}

\PY{c+c1}{\PYZsh{} y es nuestra variable dependiente}
\PY{n}{y} \PY{o}{=} \PY{n}{dataset}\PY{p}{[}\PY{l+s+s1}{\PYZsq{}}\PY{l+s+s1}{NIHSS\PYZus{}alta\PYZus{}ESTABLE\PYZus{}O\PYZus{}GRAVE}\PY{l+s+s1}{\PYZsq{}}\PY{p}{]}

\PY{c+c1}{\PYZsh{} Uso de Skicit\PYZhy{}learn para dividir datos en conjuntos de entrenamiento y prueba }
\PY{c+c1}{\PYZsh{} División 75\PYZpc{} de datos para entrenamiento, 25\PYZpc{} de datos para test}
\PY{n}{X\PYZus{}train}\PY{p}{,} \PY{n}{X\PYZus{}test}\PY{p}{,} \PY{n}{y\PYZus{}train}\PY{p}{,} \PY{n}{y\PYZus{}test} \PY{o}{=} \PY{n}{train\PYZus{}test\PYZus{}split}\PY{p}{(}\PY{n}{X}\PY{p}{,} \PY{n}{y}\PY{p}{,} \PY{n}{random\PYZus{}state}\PY{o}{=}\PY{l+m+mi}{0}\PY{p}{)}
\end{Verbatim}
\end{tcolorbox}

    \hypertarget{creaciuxf3n-del-modelo-y-entrenamiento}{%
\subsection{Creación del modelo y
entrenamiento}\label{creaciuxf3n-del-modelo-y-entrenamiento}}

Para la creación del modelo se utilizará el modelo en la forma más
estándar posible, siendo que los modelos, antes del entrenamiento,
pueden recibir ajustes para manejar los datos de entrada, de una forma u
otra. Para que sea lo más parejo posible entre modelos se dejará de
forma estándar.

    \begin{tcolorbox}[breakable, size=fbox, boxrule=1pt, pad at break*=1mm,colback=cellbackground, colframe=cellborder]
\prompt{In}{incolor}{55}{\boxspacing}
\begin{Verbatim}[commandchars=\\\{\}]
\PY{k+kn}{from} \PY{n+nn}{sklearn}\PY{n+nn}{.}\PY{n+nn}{naive\PYZus{}bayes} \PY{k+kn}{import} \PY{n}{GaussianNB}

\PY{c+c1}{\PYZsh{} Creamos el modelo de NB}
\PY{n}{nb} \PY{o}{=} \PY{n}{GaussianNB}\PY{p}{(}\PY{p}{)}
\PY{n}{nb}\PY{o}{.}\PY{n}{fit}\PY{p}{(}\PY{n}{X\PYZus{}train}\PY{p}{,} \PY{n}{y\PYZus{}train}\PY{p}{)}
\end{Verbatim}
\end{tcolorbox}

            \begin{tcolorbox}[breakable, size=fbox, boxrule=.5pt, pad at break*=1mm, opacityfill=0]
\prompt{Out}{outcolor}{55}{\boxspacing}
\begin{Verbatim}[commandchars=\\\{\}]
GaussianNB()
\end{Verbatim}
\end{tcolorbox}
        
    \hypertarget{predicciones-sobre-los-datos-de-prueba-y-muxe9tricas-de-rendimiento}{%
\subsection{Predicciones sobre los datos de prueba y métricas de
rendimiento}\label{predicciones-sobre-los-datos-de-prueba-y-muxe9tricas-de-rendimiento}}

Para llevar una forma más ordenada, es necesario crear las variables de
predicciones, para asi sacar las métricas de rendimiento más fácilmente.
Las métricas de rendimiento nos ofrecerán información de cómo se
comportó el algoritmo durante el entrenamiento, dando a conocer valores
importantes como lo son la precisión, exhaustividad, valor-F.

    \begin{tcolorbox}[breakable, size=fbox, boxrule=1pt, pad at break*=1mm,colback=cellbackground, colframe=cellborder]
\prompt{In}{incolor}{56}{\boxspacing}
\begin{Verbatim}[commandchars=\\\{\}]
\PY{c+c1}{\PYZsh{} Predicción Entrenamiento }
\PY{n}{prediccionEntreno} \PY{o}{=} \PY{n}{nb}\PY{o}{.}\PY{n}{predict}\PY{p}{(}\PY{n}{X\PYZus{}train}\PY{p}{)}

\PY{c+c1}{\PYZsh{} Predicción Tests}
\PY{n}{prediccionTests} \PY{o}{=} \PY{n}{nb}\PY{o}{.}\PY{n}{predict}\PY{p}{(}\PY{n}{X\PYZus{}test}\PY{p}{)}
\end{Verbatim}
\end{tcolorbox}

    \begin{tcolorbox}[breakable, size=fbox, boxrule=1pt, pad at break*=1mm,colback=cellbackground, colframe=cellborder]
\prompt{In}{incolor}{57}{\boxspacing}
\begin{Verbatim}[commandchars=\\\{\}]
\PY{k+kn}{from} \PY{n+nn}{sklearn} \PY{k+kn}{import} \PY{n}{metrics}

\PY{n+nb}{print}\PY{p}{(}\PY{l+s+s2}{\PYZdq{}}\PY{l+s+s2}{Entrenamiento \PYZhy{} Presición :}\PY{l+s+s2}{\PYZdq{}}\PY{p}{,} \PY{n}{metrics}\PY{o}{.}\PY{n}{accuracy\PYZus{}score}\PY{p}{(}\PY{n}{y\PYZus{}train}\PY{p}{,} \PY{n}{prediccionEntreno}\PY{p}{)}\PY{p}{)}
\PY{n+nb}{print}\PY{p}{(}\PY{l+s+s2}{\PYZdq{}}\PY{l+s+s2}{Entrenamiento \PYZhy{} Reporte de clasificación:}\PY{l+s+se}{\PYZbs{}n}\PY{l+s+s2}{\PYZdq{}}\PY{p}{,} \PY{n}{metrics}\PY{o}{.}\PY{n}{classification\PYZus{}report}\PY{p}{(}\PY{n}{y\PYZus{}train}\PY{p}{,} \PY{n}{prediccionEntreno}\PY{p}{)}\PY{p}{)}
\end{Verbatim}
\end{tcolorbox}

    \begin{Verbatim}[commandchars=\\\{\}]
Entrenamiento - Presición : 0.8529411764705882
Entrenamiento - Reporte de clasificación:
               precision    recall  f1-score   support

           0       0.80      1.00      0.89        20
           1       1.00      0.64      0.78        14

    accuracy                           0.85        34
   macro avg       0.90      0.82      0.84        34
weighted avg       0.88      0.85      0.85        34

    \end{Verbatim}

    La precisión de los datos de entrenamiento en el modelo tiene un valor
excelente de 100\% de predicción, la exhaustividad informa la cantidad
de datos capaz de identificar y en este caso es de un 100\% de los datos
y finalmente el F1 combina los valores de precisión y exhaustividad
obteniéndose un 100\% igual. Todos los valores mencionados aplican para
los estados de la variable dependiente.

    \hypertarget{matriz-de-confusiuxf3n}{%
\subsection{Matriz de Confusión}\label{matriz-de-confusiuxf3n}}

En el campo de la inteligencia artificial y en especial en el problema
de la clasificación estadística, una matriz de confusión es una
herramienta que permite la visualización del desempeño de un algoritmo
que se emplea en aprendizaje supervisado.

    \begin{tcolorbox}[breakable, size=fbox, boxrule=1pt, pad at break*=1mm,colback=cellbackground, colframe=cellborder]
\prompt{In}{incolor}{58}{\boxspacing}
\begin{Verbatim}[commandchars=\\\{\}]
\PY{k+kn}{from} \PY{n+nn}{matplotlib} \PY{k+kn}{import} \PY{n}{pyplot} \PY{k}{as} \PY{n}{plot}
\PY{k+kn}{from} \PY{n+nn}{mlxtend}\PY{n+nn}{.}\PY{n+nn}{plotting} \PY{k+kn}{import} \PY{n}{plot\PYZus{}confusion\PYZus{}matrix}
\PY{k+kn}{from} \PY{n+nn}{sklearn}\PY{n+nn}{.}\PY{n+nn}{metrics} \PY{k+kn}{import} \PY{n}{confusion\PYZus{}matrix}

\PY{n}{matriz} \PY{o}{=} \PY{n}{confusion\PYZus{}matrix}\PY{p}{(}\PY{n}{y\PYZus{}train}\PY{p}{,} \PY{n}{prediccionEntreno}\PY{p}{)}

\PY{n}{plot\PYZus{}confusion\PYZus{}matrix}\PY{p}{(}\PY{n}{conf\PYZus{}mat}\PY{o}{=}\PY{n}{matriz}\PY{p}{,} \PY{n}{figsize}\PY{o}{=}\PY{p}{(}\PY{l+m+mi}{6}\PY{p}{,}\PY{l+m+mi}{6}\PY{p}{)}\PY{p}{,} \PY{n}{show\PYZus{}normed}\PY{o}{=}\PY{k+kc}{False}\PY{p}{)}
\PY{n}{plot}\PY{o}{.}\PY{n}{tight\PYZus{}layout}\PY{p}{(}\PY{p}{)}
\end{Verbatim}
\end{tcolorbox}

\begin{center}
    	\begin{figure}[htb]
	\centering
    \adjustimage{max size={0.9\linewidth}{0.9\paperheight}}{Naive Bayes/output_91_0.png}
	\caption{Matriz de confusión de entrenamiento Naive Bayes}
	\label{fig:mcenb}
	\end{figure}
\end{center}
    
    En la matriz de confusión (1 , 1) podemos observar el resultado en el
que el modelo predice correctamente la clase positiva y en el (2, 2) el
resultado donde el modelo predice correctamente la clase negativa. Los
demás elementos de la matriz contiene valor nulo o 0, estos son los
errores de la predicción.

    \hypertarget{nauxefve-bayes---testeo-del-algoritmo}{%
\section{Naïve Bayes - Testeo del
algoritmo}\label{nauxefve-bayes---testeo-del-algoritmo}}

El testing tiene la finalidad de llevar a cabo la prueba si el modelo
funciona correctamente, identificando riesgos o erros que se produjeron
en los datos. No se realizará ajustes posteriores al testing para poder
comparar los algoritmos en la sección de resultados.

    \hypertarget{predicciones-sobre-los-datos-del-testing-y-muxe9tricas-de-rendimiento}{%
\subsection{Predicciones sobre los datos del testing y métricas de
rendimiento}\label{predicciones-sobre-los-datos-del-testing-y-muxe9tricas-de-rendimiento}}

Ahora es momento de evaluar los datos ya entrenados con el testing. Las
métricas de rendimiento nos ofrecerán información de cómo se comportó el
algoritmo durante el entrenamiento, dando a conocer valores importantes
como lo son la precisión, exhaustividad, valor-F.

    \begin{tcolorbox}[breakable, size=fbox, boxrule=1pt, pad at break*=1mm,colback=cellbackground, colframe=cellborder]
\prompt{In}{incolor}{59}{\boxspacing}
\begin{Verbatim}[commandchars=\\\{\}]
\PY{n+nb}{print}\PY{p}{(}\PY{l+s+s2}{\PYZdq{}}\PY{l+s+s2}{Tests \PYZhy{} Presición :}\PY{l+s+s2}{\PYZdq{}}\PY{p}{,} \PY{n}{metrics}\PY{o}{.}\PY{n}{accuracy\PYZus{}score}\PY{p}{(}\PY{n}{y\PYZus{}test}\PY{p}{,} \PY{n}{prediccionTests}\PY{p}{)}\PY{p}{)}
\PY{n+nb}{print}\PY{p}{(}\PY{l+s+s2}{\PYZdq{}}\PY{l+s+s2}{Tests \PYZhy{} Reporte de clasificación:}\PY{l+s+se}{\PYZbs{}n}\PY{l+s+s2}{\PYZdq{}}\PY{p}{,} \PY{n}{metrics}\PY{o}{.}\PY{n}{classification\PYZus{}report}\PY{p}{(}\PY{n}{y\PYZus{}test}\PY{p}{,} \PY{n}{prediccionTests}\PY{p}{)}\PY{p}{)}
\end{Verbatim}
\end{tcolorbox}

    \begin{Verbatim}[commandchars=\\\{\}]
Tests - Presición : 0.5833333333333334
Tests - Reporte de clasificación:
               precision    recall  f1-score   support

           0       0.55      1.00      0.71         6
           1       1.00      0.17      0.29         6

    accuracy                           0.58        12
   macro avg       0.77      0.58      0.50        12
weighted avg       0.77      0.58      0.50        12

    \end{Verbatim}

    La precisión de los datos del testing en el modelo tiene un valor de 55\% de predicción y 100\% de presicción respectivamente, la exhaustividad en el estado 0 alcanza el 100\% de los datos y en el estado 1 alcanza el 17\%.  Por otra parte, el F1 combina los valores de precisión y exhaustividad obteniéndose un 71\% en el estado 0 y un 29\% en el estado 1. 

Lo que se busca es la precisión del modelo, por consecuencia, el Algoritmo de Machine Learning Naïve Bayes tiene una precisión del 58.3\% de predicción.

    \hypertarget{matriz-de-confusiuxf3n}{%
\subsection{Matriz de Confusión}\label{matriz-de-confusiuxf3n}}

Evaluaremos la matriz de confusión que se elaboró con los datos del
testing.

    \begin{tcolorbox}[breakable, size=fbox, boxrule=1pt, pad at break*=1mm,colback=cellbackground, colframe=cellborder]
\prompt{In}{incolor}{60}{\boxspacing}
\begin{Verbatim}[commandchars=\\\{\}]
\PY{n}{matriz} \PY{o}{=} \PY{n}{confusion\PYZus{}matrix}\PY{p}{(}\PY{n}{y\PYZus{}test}\PY{p}{,} \PY{n}{prediccionTests}\PY{p}{)}

\PY{n}{plot\PYZus{}confusion\PYZus{}matrix}\PY{p}{(}\PY{n}{conf\PYZus{}mat}\PY{o}{=}\PY{n}{matriz}\PY{p}{,} \PY{n}{figsize}\PY{o}{=}\PY{p}{(}\PY{l+m+mi}{6}\PY{p}{,}\PY{l+m+mi}{6}\PY{p}{)}\PY{p}{,} \PY{n}{show\PYZus{}normed}\PY{o}{=}\PY{k+kc}{False}\PY{p}{)}
\PY{n}{plt}\PY{o}{.}\PY{n}{tight\PYZus{}layout}\PY{p}{(}\PY{p}{)}
\end{Verbatim}
\end{tcolorbox}

\begin{center}
    	\begin{figure}[htb]
	\centering
    \adjustimage{max size={0.9\linewidth}{0.9\paperheight}}{Naive Bayes/output_98_0.png}
	\caption{Matriz de confusión de testing Naive Bayes}
	\label{fig:mctnb}
	\end{figure}
\end{center}
    
    En la matriz de confusión (1 , 1) podemos observar el resultado en el
que el modelo predice correctamente la clase positiva y en el (2, 2) el
resultado donde el modelo predice correctamente la clase negativa. En el
elemento (1, 2) el modelo predice incorrectamente la clase positiva
cuando en realidad es negativa con un valor bajo y en el elemento (2 ,
1) el modelo predice incorrectamente la clase negativa cuando en
realidad es positiva.

Las afirmaciones anteriores sugieren que la las predicciones son altas,
pero también existen errores en la predicción.

    \hypertarget{nauxefve-bayes---uso-del-algoritmo}{%
\section{Naïve Bayes - Uso del
algoritmo}\label{nauxefve-bayes---uso-del-algoritmo}}

El último paso de la metodología es el uso del algoritmo, nosotros lo
utilizaremos para desarrollar probabilidades en los predictores. No
todos los modelos poseen los mismos métodos ni atributos, por ende, se
tratará de realizar comparaciones con métodos similares entre sí.

    \hypertarget{importancia-de-los-predictores}{%
\subsection{Importancia de los
predictores}\label{importancia-de-los-predictores}}

Por experiencia previa y contemplando los gráficos producidos en el paso
3, sabemos que algunas características no son útiles para nuestro
problema de predicción. Reducir la cantidad de funciones será la mejor
alternativa, lo que acotará el tiempo de ejecución, con suerte sin
comprometer significativamente el rendimiento, asi podemos examinar la
importancia de las funciones de nuestro modelo. La importancia de cada
predictor en el modelo se calcula como la reducción total (normalizada)
en el criterio de división. Si un predictor no ha sido seleccionado en
ninguna división, no se ha incluido en el modelo y por lo tanto su
importancia es 0.

    \begin{tcolorbox}[breakable, size=fbox, boxrule=1pt, pad at break*=1mm,colback=cellbackground, colframe=cellborder]
\prompt{In}{incolor}{61}{\boxspacing}
\begin{Verbatim}[commandchars=\\\{\}]
\PY{c+c1}{\PYZsh{} Predicciones probabilísticas}
\PY{c+c1}{\PYZsh{} =============================}
\PY{c+c1}{\PYZsh{} Con .predict\PYZus{}proba() se obtiene, para cada observación, la probabilidad predicha}
\PY{c+c1}{\PYZsh{} de pertenecer a cada una de las dos clases.}
\PY{n}{predicciones} \PY{o}{=} \PY{n}{nb}\PY{o}{.}\PY{n}{predict\PYZus{}proba}\PY{p}{(}\PY{n}{X\PYZus{}test}\PY{p}{)}
\PY{n}{predicciones} \PY{o}{=} \PY{n}{pd}\PY{o}{.}\PY{n}{DataFrame}\PY{p}{(}\PY{n}{predicciones}\PY{p}{,} \PY{n}{columns} \PY{o}{=} \PY{n}{nb}\PY{o}{.}\PY{n}{classes\PYZus{}}\PY{p}{)}
\PY{n}{predicciones}
\end{Verbatim}
\end{tcolorbox}

            \begin{tcolorbox}[breakable, size=fbox, boxrule=.5pt, pad at break*=1mm, opacityfill=0]
\prompt{Out}{outcolor}{61}{\boxspacing}
\begin{Verbatim}[commandchars=\\\{\}]
                0             1
0    9.996170e-01  3.830079e-04
1    9.999919e-01  8.058027e-06
2    9.999999e-01  9.046165e-08
3    9.999971e-01  2.931259e-06
4    9.999995e-01  4.810226e-07
5    9.999997e-01  3.419335e-07
6    9.991052e-01  8.948010e-04
7    9.999731e-01  2.687200e-05
8    9.997985e-01  2.015118e-04
9   9.007276e-176  1.000000e+00
10   1.000000e+00  3.399095e-08
11   1.000000e+00  0.000000e+00
\end{Verbatim}
\end{tcolorbox}
        
    Este método acepta un solo argumento que corresponde a los datos sobre
los cuales se calculan las probabilidades y devuelve una matriz de
listas que contienen las probabilidades de clase para los puntos de
datos de entrada. En este caso particular podemos observar que los
estados de la variable predictora tienen un valor de porcentaje
predictor, por ejemplo la tupla 0 posee un 99\% y fracción de precisión
para el estado 0 y un 0,11\% y fracción para el estado 1.

    \begin{tcolorbox}[breakable, size=fbox, boxrule=1pt, pad at break*=1mm,colback=cellbackground, colframe=cellborder]
\prompt{In}{incolor}{62}{\boxspacing}
\begin{Verbatim}[commandchars=\\\{\}]
\PY{c+c1}{\PYZsh{} Predicciones con clasificación final}
\PY{c+c1}{\PYZsh{} ===============================}
\PY{c+c1}{\PYZsh{} Con .predict() se obtiene, para cada observación, la clasificación predicha por}
\PY{c+c1}{\PYZsh{} el modelo. Esta clasificación se corresponde con la clase con mayor probabilidad.}
\PY{n}{predicciones} \PY{o}{=} \PY{n}{nb}\PY{o}{.}\PY{n}{predict}\PY{p}{(}\PY{n}{X\PYZus{}test}\PY{p}{)}
\PY{n}{predicciones} \PY{o}{=} \PY{n}{pd}\PY{o}{.}\PY{n}{DataFrame}\PY{p}{(}\PY{n}{predicciones}\PY{p}{)}
\PY{n}{predicciones}
\end{Verbatim}
\end{tcolorbox}

            \begin{tcolorbox}[breakable, size=fbox, boxrule=.5pt, pad at break*=1mm, opacityfill=0]
\prompt{Out}{outcolor}{62}{\boxspacing}
\begin{Verbatim}[commandchars=\\\{\}]
    0
0   0
1   0
2   0
3   0
4   0
5   0
6   0
7   0
8   0
9   1
10  0
11  0
\end{Verbatim}
\end{tcolorbox}
        
    Se observa un valor binario de 0 o 1, donde se muestra cada variable
desarrollada en el modelo puede tomar dicho valor. El valor 0 demuestra
que la tupla no logra predecir el estado 0 de la variable predictora, y
por el contrario, el estado 1 es que logra la predicción del estado en
esa tupla.


    % Add a bibliography block to the postdoc

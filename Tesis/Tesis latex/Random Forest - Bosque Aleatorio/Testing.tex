    \hypertarget{random-forest---testeo-del-algoritmo}{%
\section{Random Forest - Testeo del
algoritmo}\label{random-forest---testeo-del-algoritmo}}

El testing tiene la finalidad de llevar a cabo la prueba si el modelo
funciona correctamente, identificando riesgos o erros que se produjeron
en los datos. No se realizará ajustes posteriores al testing para poder
comparar los algoritmos en la sección de resultados.

    \hypertarget{predicciones-sobre-los-datos-del-testing-y-muxe9tricas-de-rendimiento}{%
\subsection{Predicciones sobre los datos del testing y métricas de
rendimiento}\label{predicciones-sobre-los-datos-del-testing-y-muxe9tricas-de-rendimiento}}

Ahora es momento de evaluar los datos ya entrenados con el testing. Las
métricas de rendimiento nos ofrecerán información de cómo se comportó el
algoritmo durante el entrenamiento, dando a conocer valores importantes
como lo son la precisión, exhaustividad, valor-F.

    \begin{tcolorbox}[breakable, size=fbox, boxrule=1pt, pad at break*=1mm,colback=cellbackground, colframe=cellborder]
\prompt{In}{incolor}{60}{\boxspacing}
\begin{Verbatim}[commandchars=\\\{\}]
\PY{n+nb}{print}\PY{p}{(}\PY{l+s+s2}{\PYZdq{}}\PY{l+s+s2}{Tests \PYZhy{} Presición :}\PY{l+s+s2}{\PYZdq{}}\PY{p}{,} \PY{n}{metrics}\PY{o}{.}\PY{n}{accuracy\PYZus{}score}\PY{p}{(}\PY{n}{y\PYZus{}test}\PY{p}{,} \PY{n}{prediccionTests}\PY{p}{)}\PY{p}{)}
\PY{n+nb}{print}\PY{p}{(}\PY{l+s+s2}{\PYZdq{}}\PY{l+s+s2}{Tests \PYZhy{} Reporte de clasificación:}\PY{l+s+se}{\PYZbs{}n}\PY{l+s+s2}{\PYZdq{}}\PY{p}{,} \PY{n}{metrics}\PY{o}{.}\PY{n}{classification\PYZus{}report}\PY{p}{(}\PY{n}{y\PYZus{}test}\PY{p}{,} \PY{n}{prediccionTests}\PY{p}{)}\PY{p}{)}
\end{Verbatim}
\end{tcolorbox}

    \begin{Verbatim}[commandchars=\\\{\}]
Tests - Presición : 0.6216216216216216
Tests - Reporte de clasificación:
               precision    recall  f1-score   support

           0       0.61      0.85      0.71        20
           1       0.67      0.35      0.46        17

    accuracy                           0.62        37
   macro avg       0.64      0.60      0.58        37
weighted avg       0.63      0.62      0.59        37

    \end{Verbatim}

    La precisión de los datos del testing en el modelo tiene un valor de
61\% de predicción para el estado 0 y un 67\% de predicción para el
estado 1. La exhaustividad en el estado 0 alcanza el 85\% de los datos y
en el estado 1 alcanza solo el 35\%. Por otra parte, el F1 combina los
valores de precisión y exhaustividad obteniéndose un 71\% en el estado 0
y un 46\% en el estado 1.

Lo que se busca es la precisión del modelo, por consecuencia, el
Algoritmo de ML Random Forest tiene una precisión del
62,1\% de predicción.

    \hypertarget{matriz-de-confusiuxf3n}{%
\subsection{Matriz de Confusión}\label{matriz-de-confusiuxf3n}}

Evaluaremos la matriz de confusión que se elaboró con los datos del
testing.

    \begin{tcolorbox}[breakable, size=fbox, boxrule=1pt, pad at break*=1mm,colback=cellbackground, colframe=cellborder]
\prompt{In}{incolor}{61}{\boxspacing}
\begin{Verbatim}[commandchars=\\\{\}]
\PY{n}{matriz} \PY{o}{=} \PY{n}{confusion\PYZus{}matrix}\PY{p}{(}\PY{n}{y\PYZus{}test}\PY{p}{,} \PY{n}{prediccionTests}\PY{p}{)}

\PY{n}{plot\PYZus{}confusion\PYZus{}matrix}\PY{p}{(}\PY{n}{conf\PYZus{}mat}\PY{o}{=}\PY{n}{matriz}\PY{p}{,} \PY{n}{figsize}\PY{o}{=}\PY{p}{(}\PY{l+m+mi}{6}\PY{p}{,}\PY{l+m+mi}{6}\PY{p}{)}\PY{p}{,} \PY{n}{show\PYZus{}normed}\PY{o}{=}\PY{k+kc}{False}\PY{p}{)}
\PY{n}{plt}\PY{o}{.}\PY{n}{tight\PYZus{}layout}\PY{p}{(}\PY{p}{)}
\end{Verbatim}
\end{tcolorbox}

\begin{center}
    	\begin{figure}[H]
	\centering
    \adjustimage{max size={0.5\linewidth}{0.9\paperheight}}{Random Forest - Bosque Aleatorio/output_100_0.png}
	\caption{Matriz de confusión de testing Random Forest}
	\label{fig:mctrf}
	\end{figure}
\end{center}
    
    En la matriz de confusión \ref{fig:mctrf} (1, 1) podemos observar el resultado en el que
el modelo predice correctamente la clase positiva con un alto valor y en
el (2, 2) el resultado donde el modelo predice correctamente la clase
negativa también con un alto valor, acercándose a la misma de la clase
positiva. Los demás elementos de la matriz contienen valor pequeño,
estos son los errores de la predicción.

Las afirmaciones anteriores sugieren que la las predicciones son altas,
pero también existen errores en la predicción.

    \hypertarget{importancia-de-los-predictores}{%
\subsection{Importancia de los
predictores}\label{importancia-de-los-predictores}}

Por experiencia previa y contemplando los gráficos producidos en el paso
3, sabemos que algunas características no son útiles para nuestro
problema de predicción. Reducir la cantidad de funciones será la mejor
alternativa, lo que acotará el tiempo de ejecución, con suerte sin
comprometer significativamente el rendimiento, asi podemos examinar la
importancia de las funciones de nuestro modelo. La importancia de cada
predictor en el modelo se calcula como la reducción total (normalizada)
en el criterio de división. Si un predictor no ha sido seleccionado en
ninguna división, no se ha incluido en el modelo y por lo tanto su
importancia es 0.

    \begin{tcolorbox}[breakable, size=fbox, boxrule=1pt, pad at break*=1mm,colback=cellbackground, colframe=cellborder]
\prompt{In}{incolor}{62}{\boxspacing}
\begin{Verbatim}[commandchars=\\\{\}]
\PY{c+c1}{\PYZsh{} Predicciones probabilísticas}
\PY{c+c1}{\PYZsh{} ===============================}
\PY{c+c1}{\PYZsh{} Con .predict\PYZus{}proba() se obtiene, para cada observación, la probabilidad predicha}
\PY{c+c1}{\PYZsh{} de pertenecer a cada una de las dos clases.}
\PY{n}{predicciones} \PY{o}{=} \PY{n}{rfc}\PY{o}{.}\PY{n}{predict\PYZus{}proba}\PY{p}{(}\PY{n}{X\PYZus{}test}\PY{p}{)}
\PY{n}{predicciones} \PY{o}{=} \PY{n}{pd}\PY{o}{.}\PY{n}{DataFrame}\PY{p}{(}\PY{n}{predicciones}\PY{p}{,} \PY{n}{columns} \PY{o}{=} \PY{n}{rfc}\PY{o}{.}\PY{n}{classes\PYZus{}}\PY{p}{)}
\PY{n}{predicciones}
\end{Verbatim}
\end{tcolorbox}

\begin{table}[H]
\centering
\setlength{\tabcolsep}{5pt}
\resizebox{0.2\textwidth}{!}{
\begin{tabular}{|c|l|l|}
\hline
\multicolumn{1}{|l|}{} & \multicolumn{1}{c|}{\textbf{0}} & \multicolumn{1}{c|}{\textbf{1}} \\ \hline
\textbf{0} & 0,64 & 0,36 \\ \hline
\textbf{1} & 0,74 & 0,26 \\ \hline
\textbf{2} & 0,76 & 0,24 \\ \hline
\textbf{3} & 0,74 & 0,26 \\ \hline
\textbf{4} & 0,57 & 0,43 \\ \hline
\textbf{5} & 0,62 & 0,38 \\ \hline
\textbf{6} & 0,52 & 0,48 \\ \hline
\textbf{7} & 0,65 & 0,35 \\ \hline
\textbf{8} & 0,46 & 0,54 \\ \hline
\textbf{9} & 0,32 & 0,68 \\ \hline
\textbf{10} & 0,71 & 0,29 \\ \hline
\textbf{11} & 0,75 & 0,25 \\ \hline
\textbf{12} & 0,77 & 0,23 \\ \hline
\textbf{13} & 0,67 & 0,33 \\ \hline
\textbf{14} & 0,68 & 0,32 \\ \hline
\textbf{15} & 0,7 & 0,3 \\ \hline
\textbf{16} & 0,56 & 0,44 \\ \hline
\textbf{17} & 0,37 & 0,63 \\ \hline
\textbf{18} & 0,36 & 0,64 \\ \hline
\textbf{19} & 0,57 & 0,43 \\ \hline
\textbf{20} & 0,72 & 0,28 \\ \hline
\textbf{21} & 0,57 & 0,43 \\ \hline
\textbf{22} & 0,6 & 0,4 \\ \hline
\textbf{23} & 0,64 & 0,36 \\ \hline
\textbf{24} & 0,62 & 0,38 \\ \hline
\textbf{25} & 0,5 & 0,5 \\ \hline
\textbf{26} & 0,48 & 0,52 \\ \hline
\textbf{27} & 0,43 & 0,57 \\ \hline
\textbf{28} & 0,63 & 0,37 \\ \hline
\textbf{29} & 0,52 & 0,48 \\ \hline
\textbf{30} & 0,74 & 0,26 \\ \hline
\textbf{31} & 0,48 & 0,52 \\ \hline
\textbf{32} & 0,68 & 0,32 \\ \hline
\textbf{33} & 0,61 & 0,39 \\ \hline
\textbf{34} & 0,49 & 0,51 \\ \hline
\textbf{35} & 0,57 & 0,43 \\ \hline
\textbf{36} & 0,4 & 0,6 \\ \hline
\end{tabular}%
}
\caption{Predicciones probabilísticas para cada observación Random Forest}
\label{tab:clasificacin rf}
\end{table}
        
    Este método acepta un solo argumento que corresponde a los datos sobre
los cuales se calculan las probabilidades y devuelve una matriz de
listas que contienen las probabilidades de clase para los puntos de
datos de entrada. En este caso particular podemos observar en la tabla \ref{tab:clasificacin rf} que los
estados de la variable predictora tienen un valor de porcentaje
predictor, por ejemplo, la tupla 0 posee un 64\% y fracción de precisión
para el estado 0 y un 0,36\% y fracción para el estado 1.

En síntesis se observa que cada uno de las variables reaccionan y toma
valores porcentuales a que estado del predictor pertenece.

    \begin{tcolorbox}[breakable, size=fbox, boxrule=1pt, pad at break*=1mm,colback=cellbackground, colframe=cellborder]
\prompt{In}{incolor}{63}{\boxspacing}
\begin{Verbatim}[commandchars=\\\{\}]
\PY{c+c1}{\PYZsh{} Predicciones con clasificación final}
\PY{c+c1}{\PYZsh{} ==============================================================================}
\PY{c+c1}{\PYZsh{} Con .predict() se obtiene, para cada observación, la clasificación predicha por}
\PY{c+c1}{\PYZsh{} el modelo. Esta clasificación se corresponde con la clase con mayor probabilidad.}
\PY{n}{predicciones} \PY{o}{=} \PY{n}{rfc}\PY{o}{.}\PY{n}{predict}\PY{p}{(}\PY{n}{X\PYZus{}test}\PY{p}{)}
\PY{n}{predicciones} \PY{o}{=} \PY{n}{pd}\PY{o}{.}\PY{n}{DataFrame}\PY{p}{(}\PY{n}{predicciones}\PY{p}{)}
\PY{n}{predicciones}
\end{Verbatim}
\end{tcolorbox}

\begin{table}[H]
\centering
\setlength{\tabcolsep}{10pt}
\resizebox{0.12\textwidth}{!}{
\begin{tabular}{|c|l|}
\hline
\multicolumn{1}{|l|}{} & \multicolumn{1}{c|}{\textbf{0}} \\ \hline
\textbf{0} & 0 \\ \hline
\textbf{1} & 0 \\ \hline
\textbf{2} & 0 \\ \hline
\textbf{3} & 0 \\ \hline
\textbf{4} & 0 \\ \hline
\textbf{5} & 0 \\ \hline
\textbf{6} & 0 \\ \hline
\textbf{7} & 0 \\ \hline
\textbf{8} & 1 \\ \hline
\textbf{9} & 1 \\ \hline
\textbf{10} & 0 \\ \hline
\textbf{11} & 0 \\ \hline
\textbf{12} & 0 \\ \hline
\textbf{13} & 0 \\ \hline
\textbf{14} & 0 \\ \hline
\textbf{15} & 0 \\ \hline
\textbf{16} & 0 \\ \hline
\textbf{17} & 1 \\ \hline
\textbf{18} & 1 \\ \hline
\textbf{19} & 0 \\ \hline
\textbf{20} & 0 \\ \hline
\textbf{21} & 0 \\ \hline
\textbf{22} & 0 \\ \hline
\textbf{23} & 0 \\ \hline
\textbf{24} & 0 \\ \hline
\textbf{25} & 0 \\ \hline
\textbf{26} & 1 \\ \hline
\textbf{27} & 1 \\ \hline
\textbf{28} & 0 \\ \hline
\textbf{29} & 0 \\ \hline
\textbf{30} & 0 \\ \hline
\textbf{31} & 1 \\ \hline
\textbf{32} & 0 \\ \hline
\textbf{33} & 0 \\ \hline
\textbf{34} & 1 \\ \hline
\textbf{35} & 0 \\ \hline
\textbf{36} & 1 \\ \hline
\end{tabular}%
}
\caption{Predicciones probabilísticas con clasificación final Random Forest}
\label{tab:probabilistica rf}
\end{table}
        
    Se observa en la tabla \ref{tab:probabilistica rf} un valor binario de 0 o 1, donde se muestra cada variable
desarrollada en el modelo puede tomar dicho valor. El valor 0 demuestra
que la tupla no logra predecir el estado 0 de la variable predictora, y
por el contrario, el estado 1 es que logra la predicción del estado en
esa tupla.

    \begin{tcolorbox}[breakable, size=fbox, boxrule=1pt, pad at break*=1mm,colback=cellbackground, colframe=cellborder]
\prompt{In}{incolor}{64}{\boxspacing}
\begin{Verbatim}[commandchars=\\\{\}]
\PY{n}{importancia\PYZus{}predictores} \PY{o}{=} \PY{n}{pd}\PY{o}{.}\PY{n}{DataFrame}\PY{p}{(}
                            \PY{p}{\PYZob{}}\PY{l+s+s1}{\PYZsq{}}\PY{l+s+s1}{Predictor}\PY{l+s+s1}{\PYZsq{}}\PY{p}{:} \PY{n}{dataset}\PY{o}{.}\PY{n}{drop}\PY{p}{(}\PY{n}{columns} \PY{o}{=} \PY{l+s+s2}{\PYZdq{}}\PY{l+s+s2}{NIHSS\PYZus{}alta\PYZus{}ESTABLE\PYZus{}O\PYZus{}GRAVE}\PY{l+s+s2}{\PYZdq{}}\PY{p}{)}\PY{o}{.}\PY{n}{columns}\PY{p}{,}
                             \PY{l+s+s1}{\PYZsq{}}\PY{l+s+s1}{importancia}\PY{l+s+s1}{\PYZsq{}}\PY{p}{:} \PY{n}{rfc}\PY{o}{.}\PY{n}{feature\PYZus{}importances\PYZus{}}\PY{p}{\PYZcb{}}
                            \PY{p}{)}
\PY{n+nb}{print}\PY{p}{(}\PY{l+s+s2}{\PYZdq{}}\PY{l+s+s2}{Importancia de los predictores en el modelo}\PY{l+s+s2}{\PYZdq{}}\PY{p}{)}
\PY{n+nb}{print}\PY{p}{(}\PY{l+s+s2}{\PYZdq{}}\PY{l+s+s2}{\PYZhy{}\PYZhy{}\PYZhy{}\PYZhy{}\PYZhy{}\PYZhy{}\PYZhy{}\PYZhy{}\PYZhy{}\PYZhy{}\PYZhy{}\PYZhy{}\PYZhy{}\PYZhy{}\PYZhy{}\PYZhy{}\PYZhy{}\PYZhy{}\PYZhy{}\PYZhy{}\PYZhy{}\PYZhy{}\PYZhy{}\PYZhy{}\PYZhy{}\PYZhy{}\PYZhy{}\PYZhy{}\PYZhy{}\PYZhy{}\PYZhy{}\PYZhy{}\PYZhy{}\PYZhy{}\PYZhy{}\PYZhy{}\PYZhy{}\PYZhy{}\PYZhy{}\PYZhy{}\PYZhy{}\PYZhy{}\PYZhy{}}\PY{l+s+s2}{\PYZdq{}}\PY{p}{)}
\PY{n}{importancia\PYZus{}predictores}\PY{o}{.}\PY{n}{sort\PYZus{}values}\PY{p}{(}\PY{l+s+s1}{\PYZsq{}}\PY{l+s+s1}{importancia}\PY{l+s+s1}{\PYZsq{}}\PY{p}{,} \PY{n}{ascending}\PY{o}{=}\PY{k+kc}{False}\PY{p}{)}
\end{Verbatim}
\end{tcolorbox}

\begin{table}[H]
\centering
\setlength{\tabcolsep}{5pt}
\resizebox{0.8\textwidth}{!}{
\begin{tabular}{|c|l|l|}
\hline
\multicolumn{1}{|l|}{} & \multicolumn{1}{c|}{\textbf{Predictor}} & \multicolumn{1}{c|}{\textbf{importancia}} \\ \hline
\textbf{5} & TRIGLICERIDOS & 0,149579 \\ \hline
\textbf{2} & EDAD & 0,098917 \\ \hline
\textbf{9} & NIHSS INICO ACV & 0,087893 \\ \hline
\textbf{4} & COL. TOTAL & 0,079422 \\ \hline
\textbf{3} & GLUCOSA & 0,076047 \\ \hline
\textbf{17} & GLUCOSA\_cat & 0,06392 \\ \hline
\textbf{16} & COL. TOTAL\_cat & 0,060227 \\ \hline
\textbf{10} & NIHSS alta ACV & 0,056171 \\ \hline
\textbf{6} & INR & 0,054726 \\ \hline
\textbf{7} & CONTEO G.B. & 0,052931 \\ \hline
\textbf{13} & CONTEO G.B.\_cat & 0,050685 \\ \hline
\textbf{1} & DIABETES & 0,050415 \\ \hline
\textbf{8} & GLASGOW AL INICO ACV & 0,041865 \\ \hline
\textbf{11} & NIHSS\_alta\_cat & 0,041696 \\ \hline
\textbf{22} & NIHSS\_INICIO\_cat\_Leve (Trombolisando) & 0,019722 \\ \hline
\textbf{0} & HTA & 0,010101 \\ \hline
\textbf{20} & NIHSS\_INICIO\_cat\_Déficit Mínimo & 0,005682 \\ \hline
\textbf{23} & NIHSS\_INICIO\_cat\_Moderado (Buen Pronostico) & 0 \\ \hline
\textbf{21} & NIHSS\_INICIO\_cat\_Grave & 0 \\ \hline
\textbf{12} & GLASGOW\_cat & 0 \\ \hline
\textbf{19} & NIHSS\_INICIO\_cat\_Déficit Importante & 0 \\ \hline
\textbf{18} & EDAD\_cat & 0 \\ \hline
\textbf{15} & TRIGLICERIDOS\_cat & 0 \\ \hline
\textbf{14} & INR\_cat & 0 \\ \hline
\textbf{24} & NIHSS\_INICIO\_cat\_Sin Déficit & 0 \\ \hline
\end{tabular}%
}
\caption{Importancia de los predictores Random Forest}
\label{tab:importancia predictor rf}
\end{table}
        
    \begin{tcolorbox}[breakable, size=fbox, boxrule=1pt, pad at break*=1mm,colback=cellbackground, colframe=cellborder]
\prompt{In}{incolor}{65}{\boxspacing}
\begin{Verbatim}[commandchars=\\\{\}]
\PY{c+c1}{\PYZsh{} Get numerical feature importances}
\PY{n}{importances} \PY{o}{=} \PY{n+nb}{list}\PY{p}{(}\PY{n}{rfc}\PY{o}{.}\PY{n}{feature\PYZus{}importances\PYZus{}}\PY{p}{)}
\PY{c+c1}{\PYZsh{} List of tuples with variable and importance}
\PY{n}{feature\PYZus{}importances} \PY{o}{=} \PY{p}{[}\PY{p}{(}\PY{n}{feature}\PY{p}{,} \PY{n+nb}{round}\PY{p}{(}\PY{n}{importance}\PY{p}{,} \PY{l+m+mi}{2}\PY{p}{)}\PY{p}{)} \PY{k}{for} \PY{n}{feature}\PY{p}{,} \PY{n}{importance} \PY{o+ow}{in} \PY{n+nb}{zip}\PY{p}{(}\PY{n}{feature\PYZus{}list}\PY{p}{,} \PY{n}{importances}\PY{p}{)}\PY{p}{]}
\PY{c+c1}{\PYZsh{} Sort the feature importances by most important first}
\PY{n}{feature\PYZus{}importances} \PY{o}{=} \PY{n+nb}{sorted}\PY{p}{(}\PY{n}{feature\PYZus{}importances}\PY{p}{,} \PY{n}{key} \PY{o}{=} \PY{k}{lambda} \PY{n}{x}\PY{p}{:} \PY{n}{x}\PY{p}{[}\PY{l+m+mi}{1}\PY{p}{]}\PY{p}{,} \PY{n}{reverse} \PY{o}{=} \PY{k+kc}{True}\PY{p}{)}

\PY{c+c1}{\PYZsh{} Reset style }
\PY{n}{plt}\PY{o}{.}\PY{n}{style}\PY{o}{.}\PY{n}{use}\PY{p}{(}\PY{l+s+s1}{\PYZsq{}}\PY{l+s+s1}{fivethirtyeight}\PY{l+s+s1}{\PYZsq{}}\PY{p}{)}

\PY{c+c1}{\PYZsh{} lista de x ubicaciones para trazar}
\PY{n}{x\PYZus{}values} \PY{o}{=} \PY{n+nb}{list}\PY{p}{(}\PY{n+nb}{range}\PY{p}{(}\PY{n+nb}{len}\PY{p}{(}\PY{n}{importances}\PY{p}{)}\PY{p}{)}\PY{p}{)}

\PY{c+c1}{\PYZsh{} Gráfico de barras}
\PY{n}{plt}\PY{o}{.}\PY{n}{bar}\PY{p}{(}\PY{n}{x\PYZus{}values}\PY{p}{,} \PY{n}{importances}\PY{p}{,} \PY{n}{orientation} \PY{o}{=} \PY{l+s+s1}{\PYZsq{}}\PY{l+s+s1}{vertical}\PY{l+s+s1}{\PYZsq{}}\PY{p}{,} \PY{n}{color} \PY{o}{=} \PY{l+s+s1}{\PYZsq{}}\PY{l+s+s1}{r}\PY{l+s+s1}{\PYZsq{}}\PY{p}{,} \PY{n}{edgecolor} \PY{o}{=} \PY{l+s+s1}{\PYZsq{}}\PY{l+s+s1}{k}\PY{l+s+s1}{\PYZsq{}}\PY{p}{,} \PY{n}{linewidth} \PY{o}{=} \PY{l+m+mf}{1.2}\PY{p}{)}

\PY{c+c1}{\PYZsh{} Marque las etiquetas para el eje x}
\PY{n}{plt}\PY{o}{.}\PY{n}{xticks}\PY{p}{(}\PY{n}{x\PYZus{}values}\PY{p}{,} \PY{n}{feature\PYZus{}list}\PY{p}{,} \PY{n}{rotation}\PY{o}{=}\PY{l+s+s1}{\PYZsq{}}\PY{l+s+s1}{vertical}\PY{l+s+s1}{\PYZsq{}}\PY{p}{)}

\PY{c+c1}{\PYZsh{} Etiquetas de eje y título}
\PY{n}{plt}\PY{o}{.}\PY{n}{ylabel}\PY{p}{(}\PY{l+s+s1}{\PYZsq{}}\PY{l+s+s1}{Importancia}\PY{l+s+s1}{\PYZsq{}}\PY{p}{)}\PY{p}{;} \PY{n}{plt}\PY{o}{.}\PY{n}{xlabel}\PY{p}{(}\PY{l+s+s1}{\PYZsq{}}\PY{l+s+s1}{Variable}\PY{l+s+s1}{\PYZsq{}}\PY{p}{)}\PY{p}{;} \PY{n}{plt}\PY{o}{.}\PY{n}{title}\PY{p}{(}\PY{l+s+s1}{\PYZsq{}}\PY{l+s+s1}{Importancia de Variables}\PY{l+s+s1}{\PYZsq{}}\PY{p}{)}\PY{p}{;}
\end{Verbatim}
\end{tcolorbox}

    \begin{center}
    	\begin{figure}[H]
	\centering
    \adjustimage{max size={0.9\linewidth}{0.9\paperheight}}{Random Forest - Bosque Aleatorio/output_108_0.png}
	\caption{Importancia de los predictores en Random Forest}
	\label{fig:ip}
	\end{figure}
\end{center}

    
    El cálculo de importancia muestra en la tabla \ref{tab:importancia predictor rf} y en el gráfico \ref{fig:ip} que existen muchas variables con
importancia al momento de la predicción, siendo la que destaca
``TRIGLICERIDOS'' con un 14,9\% de importancia. Viendo el grafico, se
observa que aún hay variables que no son significantes.

    \hypertarget{importancia-acumulada}{%
\subsection{Importancia acumulada}\label{importancia-acumulada}}

Ahora reduciremos la cantidad de funciones en uso por el modelo a solo
aquellas requeridas para representar el 95\% de la importancia. Se debe
usar el mismo número de características en los conjuntos de
entrenamiento y prueba.

    \begin{tcolorbox}[breakable, size=fbox, boxrule=1pt, pad at break*=1mm,colback=cellbackground, colframe=cellborder]
\prompt{In}{incolor}{66}{\boxspacing}
\begin{Verbatim}[commandchars=\\\{\}]
\PY{c+c1}{\PYZsh{} Lista de funciones ordenadas de mayor a menor importancia}
\PY{n}{sorted\PYZus{}importances} \PY{o}{=} \PY{p}{[}\PY{n}{importance}\PY{p}{[}\PY{l+m+mi}{1}\PY{p}{]} \PY{k}{for} \PY{n}{importance} \PY{o+ow}{in} \PY{n}{feature\PYZus{}importances}\PY{p}{]}
\PY{n}{sorted\PYZus{}features} \PY{o}{=} \PY{p}{[}\PY{n}{importance}\PY{p}{[}\PY{l+m+mi}{0}\PY{p}{]} \PY{k}{for} \PY{n}{importance} \PY{o+ow}{in} \PY{n}{feature\PYZus{}importances}\PY{p}{]}

\PY{c+c1}{\PYZsh{} Importancias acumulativas}
\PY{n}{cumulative\PYZus{}importances} \PY{o}{=} \PY{n}{np}\PY{o}{.}\PY{n}{cumsum}\PY{p}{(}\PY{n}{sorted\PYZus{}importances}\PY{p}{)}

\PY{c+c1}{\PYZsh{} Haz un gráfico de líneas}
\PY{n}{plt}\PY{o}{.}\PY{n}{plot}\PY{p}{(}\PY{n}{x\PYZus{}values}\PY{p}{,} \PY{n}{cumulative\PYZus{}importances}\PY{p}{,} \PY{l+s+s1}{\PYZsq{}}\PY{l+s+s1}{g\PYZhy{}}\PY{l+s+s1}{\PYZsq{}}\PY{p}{)}

\PY{c+c1}{\PYZsh{} Dibujar línea al 95\PYZpc{} de importancia retenida}
\PY{n}{plt}\PY{o}{.}\PY{n}{hlines}\PY{p}{(}\PY{n}{y} \PY{o}{=} \PY{l+m+mf}{0.95}\PY{p}{,} \PY{n}{xmin}\PY{o}{=}\PY{l+m+mi}{0}\PY{p}{,} \PY{n}{xmax}\PY{o}{=}\PY{n+nb}{len}\PY{p}{(}\PY{n}{sorted\PYZus{}importances}\PY{p}{)}\PY{p}{,} \PY{n}{color} \PY{o}{=} \PY{l+s+s1}{\PYZsq{}}\PY{l+s+s1}{r}\PY{l+s+s1}{\PYZsq{}}\PY{p}{,} \PY{n}{linestyles} \PY{o}{=} \PY{l+s+s1}{\PYZsq{}}\PY{l+s+s1}{dashed}\PY{l+s+s1}{\PYZsq{}}\PY{p}{)}

\PY{c+c1}{\PYZsh{} Formato x ticks y etiquetas}
\PY{n}{plt}\PY{o}{.}\PY{n}{xticks}\PY{p}{(}\PY{n}{x\PYZus{}values}\PY{p}{,} \PY{n}{sorted\PYZus{}features}\PY{p}{,} \PY{n}{rotation} \PY{o}{=} \PY{l+s+s1}{\PYZsq{}}\PY{l+s+s1}{vertical}\PY{l+s+s1}{\PYZsq{}}\PY{p}{)}

\PY{c+c1}{\PYZsh{} Etiquetas de eje y título}
\PY{n}{plt}\PY{o}{.}\PY{n}{xlabel}\PY{p}{(}\PY{l+s+s1}{\PYZsq{}}\PY{l+s+s1}{Variable}\PY{l+s+s1}{\PYZsq{}}\PY{p}{)}\PY{p}{;} \PY{n}{plt}\PY{o}{.}\PY{n}{ylabel}\PY{p}{(}\PY{l+s+s1}{\PYZsq{}}\PY{l+s+s1}{Importancia acumulada}\PY{l+s+s1}{\PYZsq{}}\PY{p}{)}\PY{p}{;} \PY{n}{plt}\PY{o}{.}\PY{n}{title}\PY{p}{(}\PY{l+s+s1}{\PYZsq{}}\PY{l+s+s1}{Importancia acumulada}\PY{l+s+s1}{\PYZsq{}}\PY{p}{)}\PY{p}{;}
\end{Verbatim}
\end{tcolorbox}

\begin{center}
    	\begin{figure}[H]
	\centering
    \adjustimage{max size={0.9\linewidth}{0.9\paperheight}}{Random Forest - Bosque Aleatorio/output_111_0.png}
	\caption{Importancia de los predictores acumulada en Random Forest}
	\label{fig:iparf}
	\end{figure}
\end{center}

    
    \begin{tcolorbox}[breakable, size=fbox, boxrule=1pt, pad at break*=1mm,colback=cellbackground, colframe=cellborder]
\prompt{In}{incolor}{67}{\boxspacing}
\begin{Verbatim}[commandchars=\\\{\}]
\PY{c+c1}{\PYZsh{} Encuentre el número de características para una importancia acumulada del 95\PYZpc{}}
\PY{c+c1}{\PYZsh{} Agregue 1 porque Python está indexado a cero}
\PY{n+nb}{print}\PY{p}{(}\PY{l+s+s1}{\PYZsq{}}\PY{l+s+s1}{Número de columna para el 95 }\PY{l+s+si}{\PYZpc{} d}\PY{l+s+s1}{e importancia:}\PY{l+s+s1}{\PYZsq{}}\PY{p}{,} \PY{n}{np}\PY{o}{.}\PY{n}{where}\PY{p}{(}\PY{n}{cumulative\PYZus{}importances} \PY{o}{\PYZgt{}} \PY{l+m+mf}{0.95}\PY{p}{)}\PY{p}{[}\PY{l+m+mi}{0}\PY{p}{]}\PY{p}{[}\PY{l+m+mi}{0}\PY{p}{]} \PY{o}{+} \PY{l+m+mi}{1}\PY{p}{)}
\end{Verbatim}
\end{tcolorbox}

    \begin{Verbatim}[commandchars=\\\{\}]
Número de columna para el 95 \% de importancia: 14
    \end{Verbatim}

    En el gráfico \ref{fig:iparf} la curva se dispara con la
``GLASGOW AL INICO ACV'' y se ratifica con el resultado del 95\% de
importancia medido anteriormente con el valor cercano del 95\% de
importancia acumulada de los datos.


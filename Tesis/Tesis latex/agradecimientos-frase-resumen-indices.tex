%----Frase motivadora -----------------------------------
\newpage{\ }
\pagenumbering{roman}
\setcounter{page}{1}
\thispagestyle{empty}
\vfill
\begin{flushright}
	\emph{"Nunca olvides que basta una persona o una idea para cambiar tu vida para siempre, ya sea para bien o para mal"}\\
	\textbf{\textit{J Brown}}
\end{flushright}
\vfill

%- Agradecimientos -------------------------------------
\addcontentsline{toc}{chapter}{AGRADECIMIENTOS}
\chapter*{Agradecimientos}
\spacing{1.5}

A Dios en primera instancia, que ha sido bueno conmigo durante toda mi vida, agradezco lo que hizo, lo que está haciendo y lo que hará en mí. Gracias por permitirme sonreír frente a todas las personas y adquirir virtudes de todas ellas, haciendo que crezca como ser humano y me expanda de distintas formas.\\
\par A mi profesora guía Carola Figueroa Flores, Doctora en Informática con Mención Cum Loude. Sus concejos, ideas fueron siempre útiles cuando mis soluciones a mis problemas se volvían confusas. Gracias por sus palabras de aliento, historias de experiencias vividas y sus orientaciones.\\
\par A mis profesores, quienes con sus conocimientos formaron al profesional que soy actualmente. Gracias por su paciencia, dedicación y tolerancia, haciendo una mención especial a mi Jefa de Carrera Marlene Muñoz Sepúlveda que estuvo siempre brindándome apoyo y consejo.\\
\par A mi madre, quien siempre me ha dado un amor inconmensurable, todo lo que podía dar y más y esperanza en todas las cosas que me esperan a futuro. Gozoso de tenerte como madre y que me acompañe en todos mis momentos. Gracias por ser quien eres y por creer en mí.\\
\par A mis amigos y compañeros, que dentro de esta maravillosa aventura dimos nuestro mejor esfuerzo para cumplir con todas las exigencias de la carrera y cooperando mutuamente para lograr el mejor resultado para todos. Especialmente, Diego Garrido y Daniel Gonzáles por su apoyo, constancia y amistad, que con ustedes la carrera fue menos pesada, más divertida y con un largo historial de aventuras que estuvieron en ella. Gracias por estar allí siempre.


%- Resumen --------------------------------------------
\addcontentsline{toc}{chapter}{RESUMEN}
\chapter*{Resumen}
\spacing{1.5}

\lettrine[lines=4, slope=0.1em, findent=0.2em, nindent=0.6em]{L} os Accidentes Cerebro Vascular, también llamados ACV o ictus, son de las principales causas de muerte en hombres y mujeres en Chile y el mundo. Los ACV podemos encontrarlos de dos tipos: ACV Isquémico por obstrucción de un vaso sanguíneo y ACV Hemorrágico por rotura de un vaso sanguíneo.\\
\par El proyecto utiliza una de las técnicas más conocidas de la Inteligencia Artificial (IA), como lo es el aprendizaje automático (Machine Learning) para realizar clasificaciones sobre pacientes que han sufrido un ACV y poder tomar decisiones prematuramente gracias a los modelos predictivos. La recopilación de datos de los pacientes fue una colaboración entre la investigadora e Informática Dra. Carola Figueroa con el médico e investigador Dr. Carlos Escudero que obtuvieron los datos del Hospital Herminda Martin de Chillán.\\
\par Se plantea la posibilidad de clasificar al paciente para predecir si tendrá un buen pronóstico cuando este sea dado de alta, por medio de la escala internacional de NIHSS (Escala de Accidentes Cerebrovasculares de los Institutos Nacionales de Salud), escala que mide el daño neurológico en los pacientes.\\
\par Como veremos a continuación, existen muchos tipos de algoritmos de Machine Learning, pero hay algunos que se repiten en el área de salud. Escogeremos 4 algoritmos y los desarrollaremos lo más simple posible, para que se pueden comparar con las mismas métricas y ningún algoritmo sufra una ventaja significativa sobre otro.


% Keywords command
\providecommand{\pclaves}[1]
{
  \small	
  \textbf{\textit{Palabras Claves: }} #1
}

\begin{pclaves}
	{Accidente Cerebro Vascular, Machine Learning, Predicción}
\end{pclaves}


%- Abstract ------------------------------------------
\addcontentsline{toc}{chapter}{ABSTRACT}
\chapter*{Abstract}
\spacing{1.5}

\lettrine[lines=4, slope=0.1em, findent=0.2em, nindent=0.6em]{C}erebro Vascular Accident, also called ACV or stroke, are one of the main causes of death in men and women in Chile and the world. CVA can be found in two types: Ischemic CVA due to obstruction of a blood vessel and Hemorrhagic CVA due to rupture of a blood vessel.\\
\par The project uses one of the best-known Artificial Intelligence (AI) techniques, such as Machine Learning, to classify patients who have suffered a stroke and to make premature decisions thanks to predictive models. The collection of patient data was a collaboration between the researcher and IT Dr. Carola Figueroa with the physician and researcher Dr. Carlos Escudero who obtained the data from the Herminda Martin de Chillán Hospital.\\
\par The possibility of classifying the patient to predict that he will have a good prognosis when he is discharged is raised, using the international scale of NIHSS (Cerebrovascular Accident Scale of the National Institutes of Health), a scale that measures neurological damage in patients. patients.\\
\par As we will see below, there are many types of Machine Learning algorithms, but there are some that are repeated in the health area. We will choose 4 algorithms and we will develop them as simple as possible, so that they can be compared with the same metrics and no algorithm suffers a significant advantage over another.\\

% Keywords command
\providecommand{\keywords}[1]
{
  \small	
  \textbf{\textit{Keywords: }} #1
}

\begin{keywords}
	{Stroke, Machine Learning, Prediction}
\end{keywords}



	%indice
	\tableofcontents
	\listoffigures
	\listoftables
%%%%%%%%%%%%%%%%%%%%%%%%%%%%%%%%%%%%%%%

\newpage{\ }
\thispagestyle{empty}
\newpage
\pagenumbering{arabic}
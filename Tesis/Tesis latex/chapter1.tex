\doublespacing
\chapter{INTRODUCCIÓN}
\spacing{1.5}
\lettrine[lines=4, slope=0.1em, findent=0.2em, nindent=0.6em]{E}{n} nuestros tiempos los computadores o similares son indispensables en las tareas del día a día, ayudando a mejorar la calidad de los trabajos, provocando un gran impacto en la sociedad hoy más que nunca. Las múltiples ramas de las ciencias de la computación han ayudado al desarrollo progresista de diversas áreas de trabajo e investigación, como lo es medicina, economía, estadística, entre otras. Al tener un gran número de datos disponibles, esto permite combinar y utilizar esta información de diferentes maneras, aumentando así las posibilidades de combinación entre las áreas. Esto a su vez puede mejorar la precisión y eficacia de los modelos en IA en tareas específicas.\\
\par Durante las últimas décadas se ha incrementado enormemente nuestra capacidad para recoger información en cualquier actividad, por ejemplo, en los negocios, recopilando información sobre procesos de producción, ventas, servicios, campañas de marketing, entre varios. La gran cantidad de información ha crecido el interés de las personas por utilizarlos, para extraer la información que represente una ventaja competitiva, lo que es particularmente relevante para las organizaciones.\\ 
\par El campo interdisciplinario de Ciencia de la Computación involucra métodos científicos, procesos y sistemas para extraer conocimiento o mejor entendimiento de datos en sus diferentes formas, ya sean información útil o no útil, idealmente estructurado. Dentro de Ciencias de la Computación, existen especialidades que solucionan un problema específico, de las cuales nosotros nos enfocaremos en la Inteligencia Artificial, que posee una variedad de metodologías para resolver los problemas de forma eficaz y eficiente. \\
\par Actualmente la IA ha experimentado un rápido crecimiento en los últimos años debido a su capacidad para resolver problemas con una alta precisión y su aplicabilidad en una amplia gama de disciplinas y áreas de trabajo. Muchos sectores, incluyendo el comercio, la salud y la tecnología social, han adoptado herramientas basadas en IA para mejorar su eficiencia y productividad. Esto ha llevado a un aumento en el mercado de productos y servicios relacionados con la IA, y se espera que esta tendencia continúe en el futuro.\\
\par Una de las principales ramas de la Inteligencia Artificial es el Machine Learning (ML), destacándose en áreas como análisis de datos, minería de datos, reconocimiento de patrones, entre varias. ML, constituye una de las ramas más atractivas para los sectores que trabajan con grandes cantidades de datos.\\
\par Este trabajo se abordará una problemática referente al ACV Isquémico,  siendo una de las principales causas de muerte a nivel mundial y específicamente dentro de nuestra región es la segunda causa de muerte. La finalidad es analizar algunos factores que se producen al evaluar un ACV con ML.\\


\doublespacing
\section{Descripción del Problema}
\spacing{1.5}
El ACV, hoy en día, son una de las principales causas de muerte, asimismo las personas que logran sobrevivir quedan con secuelas o discapacidades en la mayoría de los casos, presentándose con más frecuencia, sorpresivamente, en adultos jóvenes, pero también está aumentando en los adultos mayores \cite{Ortiz-Galeano2020}. \\
\par En Chile, solo el año 2021 hubo 29.542 egresos hospitalarios por ACV, siendo la segunda causa de mortalidad a nivel país, después de las enfermedades isquémicas del corazón y no considerando la pandemia por el SARS-CoV-2. El registro de defunciones por ACV llegó a los 7.501 casos ese mismo año, lo que equivale a una muerte cada 1 hora y 12 minutos \cite{Minsal2022}.\\
\par EN la salud es vital disminuir las posibles de enfermedades en el menor plazo posible, vinculado a esto, el ACV posee una alta tasa de morbimortalidad, es decir, el ACV en sí o por las secuelas las muertes son muy rápidas en poco tiempo \cite{Gaudiano2019}. El impacto negativo de las secuelas en los pacientes tiene un alto costo sanitario en lo intrahospitalario como extra hospitalario, en lo físico la perdida de la movilidad de una parte de su cuerpo (discapacidad) y en lo social, quizás la pérdida de un sentido (hablar o escuchar) o disminución de la calidad de vida de la persona. Ante lo anteriormente expuesto, la aparición de la enfermedad y su posible evolución se debería a los factores de riesgos para esta enfermedad \cite{Cabrera2020}.\\
\par En Ñuble, se cuenta con una red hospitalaria que lleva un registro de las enfermedades y pacientes que ingresan a los hospitales, de este modo se determinó que una de las principales causas de muerte en la región es consecuencia de los ACV. En términos de probabilidades, el ACV representa una persona muerta al dia, esto la hace la primera causa de muerte de la zona y una de las más importantes del pais, ya que el escenario se repite para las demás zonas. Esta enfermedad se trata de una urgencia, donde acceder a un tratamiento oportunamente, puede establecer la diferencia en el pronóstico de salud. Las acciones deben estar destinadas a preservar la integridad del tejido cerebral \cite{ServicioSaludNuble}.\\
\par Para obtener resultados óptimos, el médico debe realizar una serie de exámenes,  los cuales  ayudarían a obtener  un pronóstico más  certero y  lograr manejar  adecuadamente el estado de salud del paciente. La IA y el ML  aporta a la predicción del diagnóstico a través  de la obtención de resultados en un tiempo acotado,  además de lograr una mayor acertividad de este.  Es por esta razón que el rol de la IA jugará un papel importante  en el  futuro de la salud,  esto se puede observar  en los paises más desarrollados   los cuales  presentan estudios exitosos   de  exploración y ejecución en  esta área. Al mismo tiempo, los errores de diagnósticos en los países en vías de desarrollo se ven incrementados, lo que nos alerta de emplear  lo antes posible  herramientas ligadas a la IA, esperando lograr  con estas  una mayor  acertibidad  en el diagnóstico de enfermedades y  perfeccionar la medicina preventiva \cite{Curioso2020}.


\doublespacing
\section{Descripción del proyecto}
\spacing{1.5}
La presente sección tiene como propósito dar a conocer la hipótesis y los objetivos del proyecto.\\

\doublespacing
\subsection{Hipótesis}
\spacing{1.5}
Es posible determinar la técnica de ML más precisa, con el fin de clasificar según el tipo de secuela a los pacientes que han sufrido ACV Isquémico del Hospital Herminda Martín\\

\doublespacing
\subsection{Objetivo General}
\spacing{1.5}
Detectar la mejor técnica de ML para identificar los factores de mal pronóstico en adultos con diagnóstico de ACV Isquémico a través del análisis comparativo de su precisión para pacientes del Hospital Herminda Martín.\\

\doublespacing
\subsection{Objetivos Específicos}
\spacing{1.5}
\begin{enumerate}
	\renewcommand{\theenumi}{\Roman{enumi}} %Números romanos en mayúscula
	\item Estudiar las distintas técnicas de ML que existen en la literatura. 
	\item Determinar qué técnicas de ML se van a comparar.
	\item Elegir un mecanismo adecuado, para aplicar un modelo de ML  al problema.
	\item Implementar las técnicas seleccionadas de ML, para identificar la similitud entre las variables clínicas en los diferentes grupos y  las variables que tienen mayor importancia en el pronóstico de los usuarios con diagnóstico ACV isquémico.
	\item Realizar una comparación de las distintas técnicas de ML, en base a su precisión.
\end{enumerate}
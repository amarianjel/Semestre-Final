\doublespacing
\chapter{INTRODUCCIÓN}
\spacing{1.5}
\lettrine[lines=4, slope=0.1em, findent=0.2em, nindent=0.6em]{E}{n} nuestros tiempos los computadores o similares son indispensables para nuestro día a día, ayudando a mejorar la calidad de los trabajos y llevándolos a una mejor producción, siendo un gran impacto en la sociedad hoy más que nunca. Las múltiples ramas de las ciencias de la computación han ayudado al desarrollo progresista de otras áreas de trabajo e investigación, como lo es la medicina, economía, estadística o social, todo lo anterior mencionado se debe a la gran cantidad de bancos de memoria que puede procesarse y su rapidez, esto hace que las posibilidades de combinación sean altísimas entre las áreas.\\
\par Durante las últimas décadas se ha incrementado enormemente nuestra capacidad para recoger información en cualquier actividad, por ejemplo, en los negocios, recopilando información sobre procesos de producción, ventas, servicios, campañas de marketing, entre varios. La gran cantidad de información ha crecido el interés de las personas por utilizarlos para extraer la información que represente una ventaja competitiva para las organizaciones.\\ 
\par El campo interdisciplinario de la ciencia de la computación involucra métodos científicos, procesos y sistemas para extraer conocimiento o mejor entendimiento de datos en sus diferentes formas, ya sean información útil o no útil, idealmente estructurado. Dentro de las ciencias de la computación, existen especialidades que solucionan un problema específico, de las cuales nosotros nos enfocaremos en la Inteligencia Artificial, que posee una variedad de metodologías para resolver los problemas de forma eficaz y eficiente. \\
\par Actualmente la IA se ha disparado en ascenso, ya que nos ayuda a resolver problemas con mucha certeza (aplicando bien el método) y es aplicable para un gran número de disciplinas de áreas de trabajo. El mercado se ha disparado con herramientas que utilizan aplicaciones con Inteligencia Artificial, en opciones comerciales, salud, social, etc.\\
\par Procesando los datos con Inteligencia Artificial llegamos a una de sus principales ramas y una de las más conocidas que es el Machine Learning (ML), tomando un rol importantísimo en tareas con aprendizaje, análisis de datos, minería de datos, reconocimiento de patrones, entre varias. Todo lo anterior mencionado del ML, lo vuelve una de las ramas más atractivas para los sectores que trabajan grandes cantidades de datos.\\
\par Este trabajo se abordará una problemática referente al área de la salud, en una de las principales causas de muerte a nivel mundial y específicamente dentro de nuestra región es la segunda causa de muerte. La finalidad es analizar algunos factores que se producen al evaluar un ACV con ML.\\


\doublespacing
\section{Descripción del Problema}
\spacing{1.5}
Los ACV, hoy en día es una de las principales causas de muerte, asimismo las personas que logran sobrevivir quedan con secuelas o discapacidades en la mayoría de los casos, presentándose con más frecuencia sorpresivamente en los adultos jóvenes, pero también aumentando en los adultos mayores \cite{Ortiz-Galeano2020}. \\
\par En Chile, los ACV son unos de las principales causas de muerte, solo el año 2021 hubieron 29.542 egresos hospitalarios por ACV siendo la segunda causa de mortalidad a nivel país, después de las enfermedades isquémicas del corazón y no considerando la pandemia por el SARS-CoV-2. El registro de defunciones por ACV llego a los 7.501 casos ese mismo año, lo que equivale a 1 muerte cada 1 hora y 12 minuto \cite{Minsal2022}.\\
\par La salud es vital para estimar nuestro bienestar y disminuir las posibles enfermedades en el menor plazo posible, vinculado a esto, el ACV posee una alta tasa de morbimortalidad, las muertes por esta enfermedad son muy rápidas en poco tiempo, una vez detectada por el ACV en sí o por las secuelas \cite{Gaudiano2019}. El impacto negativo de las secuelas en los pacientes es un alto costo sanitario, físico y social, encontrando en lo sanitario lo intrahospitalario como extra hospitalario, en lo físico la perdida de la movilidad de una parte de su cuerpo (discapacidad) y en lo social, quizás la pérdida de un sentido (hablar o escuchar) o disminución de la calidad de vida de la persona. Ante lo anteriormente expuesto, la aparición de la enfermedad y su posible evolución se debería a los factores de riesgos para esta enfermedad \cite{Cabrera2020}.\\
\par En Ñuble, se cuenta con una red hospitalaria que lleva un registro de las enfermedades y pacientes que ingresan a los hospitales, de este modo se determinó que una de las principales causas de muerte en la región es consecuencia de los ACV, además una persona muerta al día y representa la primera causa en la zona y el país, hablando en términos de probabilidades. Esta enfermedad se trata de una urgencia, donde acceder a un tratamiento oportunamente, puede establecer la diferencia en el pronóstico de salud. Las acciones deben estar destinadas a preservar la integridad del tejido cerebral \cite{ServicioSaludNuble}.\\
\par Para obtener resultados más óptimos, los médicos deben realizar una serie de exámenes para ayudar al pronóstico del paciente, ayudando a conservar la mayor parte de salud en la persona. Para una predicción médica con más apoyo, la IA y el ML ayuda a obtener resultados en el menor tiempo posible, lo que gracias a exámenes que se deben llevar a cabo. Es por esta razón que el rol de la IA jugará un papel importante a futuro en la salud, es así como esta disciplina está explorándose y ejecutándose en los países más desarrollados. Al mismo tiempo, los errores de diagnósticos en los países en vías de desarrollo, las herramientas ligadas a la IA pueden cumplir con el rol de optimizar el diagnóstico de enfermedades y ayudar a la medicina preventiva \cite{Curioso2020}.\\


\doublespacing
\section{Descripción del proyecto}
\spacing{1.5}
La presente sección tiene como propósito dar a conocer la hipótesis y los objetivos del proyecto.\\

\doublespacing
\subsection{Hipótesis}
\spacing{1.5}
Es posible determinar la técnica de ML más precisa, con el fin de clasificar según el tipo de secuela a los pacientes que han sufrido ACV Isquémico del Hospital Herminda Martín\\

\doublespacing
\subsection{Objetivo General}
\spacing{1.5}
Detectar la mejor técnica de ML para identificar los factores de mal pronóstico en adultos con diagnóstico de ACV Isquémico a través del análisis comparativo de su precisión para pacientes del Hospital Herminda Martín.\\

\doublespacing
\subsection{Objetivos Específicos}
\spacing{1.5}
\begin{enumerate}
	\renewcommand{\theenumi}{\Roman{enumi}} %Números romanos en mayúscula
	\item Estudiar las distintas técnicas de ML que existen en la literatura. 
	\item Determinar qué técnicas de ML se van a comparar.
	\item Elegir un mecanismo adecuado, para aplicar un modelo de ML  al problema.
	\item Implementar las técnicas seleccionadas de ML, para identificar la similitud entre las variables clínicas en los diferentes grupos y  las variables que tienen mayor importancia en el pronóstico de los usuarios con diagnóstico ACV isquémico.
	\item Realizar una comparación de las distintas técnicas de ML, en base a su precisión.
\end{enumerate}
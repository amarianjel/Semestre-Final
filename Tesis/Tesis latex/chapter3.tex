\doublespacing
\chapter{TRABAJOS RELACIONADOS}
\label{sec:trabajos relacionados}
\spacing{1.5}
\lettrine[lines=4, slope=0.2em, findent=0.2em, nindent=0.6em]{E}{l}  análisis del estado del Arte contempla una revisión bibliográfica, contextualizando los avances en la investigación acerca de la IA y el área de salud.\\

\par Los avances en hospitales de países desarrollados han permitido la implementación de la IA en sus sistemas, lo que implica nuevos desafíos , como lo son el procesamiento de grandes volúmenes de información,  datos incompletos debido a la incompatibilidad de los sistemas en que se registran o incluso la presión de entregar una investigación o producto sin prolijidad y beneficio real para las personas. \cite{Nagendran2020} Google es un ejemplo claro de esta situación donde  sus algoritmos con IA presentan los problemas mencianos anteriormente.\\

\par Con el aumento de las tecnologías en el campo médico y avance de las técnicas de ML, se ha desarrollado un interés por el mundo científico para predecir algún proceso anticipado o secuela después de un ACV. Investigadores en el 2019 \cite{Heo2019}, realizaron estudio utilizando Random Forest, Redes Neuronales y Logistic Regression para predecir la prognosis de un paciente de ACV Isquémico tres meses después del evento inicial. Los autores deseaban predecir la mortalidad a los 3 años luego de salir de la rehabilitación con un algoritmo basado en Decision Tree. El mejor modelo fue Random Forest con la implementación del minority oversampling technique, el cuál logró un nivel de predicción del modelo de 0.928  \cite{Scrutinio2020}. Yu et al. \cite{Yu2020} usaron técnicas de ML considerando el Decision Tree, siendo el objetivo clasificar la severidad del ACV Isquémico. El árbol se construyó originalmente con 13 variables, de 18 propuestas, los datos usados fueron de personas mayores de 65 años del National Institutes of Health Stroke Scale. Con esta técnica se logró tener un accuracy del 91.11\%. prediciendo el nivel de discapacidad, dentro de las 24 horas  y posteriormente a los 90 días  \cite{Xie2018}. Los predictores incluían información de exámenes de escáner, demografía e información clínica de los pacientes.\\

\par Con la herramienta de las imágenes nace la posibilidad de generar algoritmos para un método más actualizado en la detección de esta enfermedad y su futura prevención, puesto que existen variadas técnicas \cite{Wintermark2013} como la RM con DWI para la evaluación de presencia y extensión de isquemia posterior y la CTA y DSA para la trombosis de arteria. Continuando  con las investigaciónes en imágenes, un estudio en Estados Unidos \cite{Garcia2019}  analizó 610.000 casos y 185.000 con recurrencia en el año 2019,  los cuales mostraban lesiones visibles en sus exámenes de imagenología, con dicha información se pudo sustentar que el manejo médico y la prevención secundaria son vitales para mermar las secuelas de los pacientes. Además de la necesidad de educar a la comunidad para reconocer algunos síntomas del ACV y así acudir al centro médico más cercano. Con base en el estudio anteriormente señalado podemos predecir que esta enfermedad llegará a un 6,2\% de la población en  países desarrollados, por esta razón es importante crear un modelo que nos permita predecir las secuelas o futuros problemas en el tratamiento, llevando a cabo una prevención exitosa, evitando la recurrencia. Por lo anteriormente expuesto es conveniente que el algoritmo para la atención deba estar basado en experiencias nacionles e internacionales \cite{Garcia2019}\\

\par La IA puede generar investigaciones DL relacionados a los estudios de Aprendizaje Profundo como estudio de revisión sistemática del diseño, estándares de informes y afirmaciones de los estudios de aprendizaje profundo \cite{Pang2017}. El objetivo de un modelo con DL es que logre ser  efectivo y eficaz y para ello es necesario trabajar con médicos expertos, a fin de que puedan  evaluar los diagnósticos mediante imágenes y contrastar los resultados obtenidos con la IA. Esta investigación posee una gran cantidad de datos como Ensayos controlados, datos Medline y ensayos que la Organización Mundial de la Salud posee desde el 2010 hasta el 2019, de los cuales se encontraron registros aleatorios de aprendizaje profundo con bajo nivel de sesgo. Por consiguiente la información que es emitida por los modelos de aprendizaje profundo puede ser manipulada por los expertos, ya que los algoritmos \cite{Nagendran2020} son experimentales y aunque son pioneros en la materia no pueden dejar en total confianza al algoritmo para que determine un diagnóstico más certero.\\

\par Las redes de datos convolucionales demostraron en Corea que el uso de ellas es una herramienta que predice con precisión los cuidados intensivos en servicios médicos (Kang et al., 2020), asumiendo que el modelo predictivo, basado en DL, es superior a las otras herramientas de predicción y puntuaciones convencionales \cite{Bioetica2022}. Cabe destacar, que el algoritmo de aprendizaje es muy eficiente por la cantidad de capas que puede poseer el modelo, puesto que entre más capas mayor puede ser el aprendizaje. Pese a las evidencias que demuestran de la efectividad y el desempeño del modelo, pueda ser mejor al del humano, existe un miedo por la implementación en los sistemas de salud, lo que puede provocar que el crecimiento del DL, perezca de una base amplia para su desarrollo \cite{Nagendran2020}.\\

\par En el área de la implementación de un modelo con CNN, encontramos el trabajo de Chunjiao Dong , Chunfu Shao,1,2 Juan Li, and Zhihua Xiong del 2018, que desarrolla específicamente la predicción sobre los accidentes de tránsito \cite{shao2018improved}. Ellos demuestran una técnica novedosa con un modelo de regresión multivariable, que presenta la relación entre lo examinado y los accidentes de tránsitos. Como resultados el módulo identifica las variables de entrada y representaciones de características de salida, aunque se haya reducido su magnitud, se conserva la información original. Además, el modelo propuesto explica mejor los problemas de heterogeneidad en predicción de accidentes de tráfico y puede ser aplicado a casos similares.\\

\par En este caso el modelo propuesto en contraste con el SVM en la categoría choque con daños menores es significativo (29.961\% versus 61.350\%), así es como la predicción medida por el RMSD se puede mejorar un  84,58\% y un 158,27\% en comparación con el modelo de aprendizaje profundo sin la capa de regresión y el modelo SVM.\\





\par El trabajo “Intelligence versus clinicians: systematic review of design, reporting standards, and claims of deep learning studies” del año 2020 \cite{Nagendran2020}, utilizó el Deep Learning con Redes Neuronales Convolucionales, tuvo como objetivo examinar sistemáticamente el diseño, los estándares de informes, el riesgo de sesgo y las afirmaciones de los estudios que comparan el rendimiento de los algoritmos de aprendizaje profundo de diagnóstico para imágenes médicas con el de médicos expertos. La investigación evalúo mediante estándares consolidados de informes de ensayo para estudios aleatorios informe transparente de un modelo de predicción multivariable para pronóstico o diagnóstico individual para estudios no aleatorios, registros de ensayos aleatorios y estudios no aleatorios que comparan el rendimiento en imágenes médicas con un grupo contemporáneo de uno o más médicos expertos. Estos estudios seleccionados tenían como objetivo utilizar imágenes médicas para predecir el riesgo absoluto de enfermedad existente o la clasificación en grupos de diagnóstico (p. ej., enfermedad o no enfermedad). El estudio concluyó que hay una escasez de estudios prospectivos de DL y ensayos aleatorios en el campo de las imágenes médicas. La mayoría de los ensayos no aleatorios no fueron prospectivos, tuvieron un alto riesgo de sesgo y se desviaron de los estándares de información existentes. La mayoría de los estudios carecen de disponibilidad y código de datos, y los grupos de comparación humanos suelen ser pequeños. Otros estudios deberían reducir el riesgo de sesgo, mejorar la importancia clínica en el mundo real, mejorar los informes y la transparencia y corregir conclusiones moderadas.\\

\par El trabajo “Machine Learning–based model for prediction of outcomes in acute stroke” del año 2019 \cite{Heo2019}, utilizó Random Forest, Redes Neuronales y Logistic Regression para la predicción de prognosis de un paciente de ACV Isquémico tres meses después del evento inicial. El objetivo de este trabajo era buscar el mejor algoritmo para la problemática planteada.
El estudio demostró que los algoritmos de ML, en particular la Red Neuronal Profunda, pueden mejorar la predicción de resultados a largo plazo para pacientes con ACV isquémico.\\

\par En “Machine learning to predict mortality after rehabilitation among patients with severe stroke” del año 2020 \cite{Scrutinio2020}, utilizó Logistic Regression y Random Forest con y sin implementación SMOTE (técnica estadística de sobremuestreo de minorías sintéticas para aumentar el número de casos de un conjunto de datos de forma equilibrada) para predecir la mortalidad después de la rehabilitación entre pacientes con ACV grave. El objetivo de este estudio era doble: evaluar el rendimiento relativo de los algoritmos basados en ML, con o sin la aplicación SMOTE, para predecir la mortalidad a largo plazo en pacientes con ACV con discapacidad grave y comparar el rendimiento de los algoritmos de ML con el de un modelo de Logistic Regression estándar. El estudio demostró que los algoritmos de ML superaron al modelo Logístico estándar, para predecir la mortalidad a los 3 años, además después de la implementación de SMOTE, los algoritmos de ML exhibieron un rendimiento general excelente, superando a los algoritmos sin la aplicación SMOTE, si bien las diferencias fueron pequeñas, el algoritmo RF exhibió el mejor rendimiento entre los algoritmos SMOTE.\\

\par La investigación “An elderly health monitoring system using machine learning and in-depth analysis techniques on the nihss stroke scale” del año 2020 \cite{Yu2020}, utilizo Random Forest, Decision Tree, Logistic Regression y Artificial Neural Networks, propone un nuevo sistema de predicción y análisis en profundidad de la gravedad del ACV en personas mayores de 65 años basado en la escala de ACV de NIHSS y el mejor algoritmo de ML que es aplicable a la escala. Como conclusiones el sistema clasifica y analiza de forma automática la gravedad de la apoplejía en cuatro clases que se utilizaron como clasificación, utilizando las funciones NIHSS recopiladas en tiempo real. También el sistema proporciona a los pacientes y sus familias información de alarma sobre la gravedad del ACV en tiempo real, para que los pacientes puedan recibir visitas al centro médico y atención de emergencia. Con Decision Tree se realizó un análisis semántico con reglas adicionales destalladas.\\

\par En “Use of gradient boosting machine learning to predict patient outcome in acute ischemic stroke on the basis of imaging, demographic, and clinical information” del año 2018 \cite{Xie2018}, utilizó Decision Tree con aumento de gradiente (GBM) y refuerzo de gradiente extremo (XGB). El objetivo de este estudio fue integrar biomarcadores comunes de ACV utilizando métodos de ML y predecir el resultado de la recuperación del paciente a los 90 días. El estudio concluyo que los GBM basados en Decision Tree pueden predecir el resultado de la recuperación de los pacientes con ACV al ingreso con un AUC alto. Dividir los grupos de pacientes sobre la base de la recanalización y la no recanalización puede ayudar potencialmente con el proceso de decisión del tratamiento.\\

\par El trabajo “Imaging recommendations for acute stroke and transient ischemic attack patients: A joint statement by the american society of neuroradiology, the american college of radiology, and the society of neurointerventional surgery” del año 2013 \cite{Wintermark2013}, utilizo NCCT que es una técnica estándar de diagnóstico por imágenes aceptada para la exclusión de hemorragia intracraneal y se ha incorporado en los criterios de inclusión en ensayos clínicos aleatorizados, llevando a su uso generalizado continuado en imágenes de ACV  agudos. Como resumen en pacientes con ACV agudo que son candidatos para trombólisis IV, se recomiendan imágenes de NCCT para excluir hemorragia intracraneal y determinar la extensión de los cambios isquémicos, además los resultados concordantes de al menos 2 técnicas de imagen no invasivas se pueden usar para determinar la elegibilidad del tratamiento para los procedimientos de revascularización.\\

\par En “Actualización en diagnóstico y tratamiento del ataque cerebrovascular isquémico agudo” del año 2019 \cite{Garcia2019}, tuvo como objetivo de esta revisión es presentar una actualización sobre los métodos diagnósticos actuales  y  las  distintas  terapias  disponibles según  sea  el  caso  de  cada  paciente,  para  el ACV isquémico agudo, con un enfoque clínico práctico, ordenado y aplicable al escenario actual de salud en Colombia. Como conclusión, los pacientes que son candidatos a un tipo de terapia especifica post ACV deben regirse con algunos criterios como escala de Ranking, NIHSS, entre otros, que señala el algoritmo de árbol de decisiones y es importante contar con políticas en salud pública enfocadas en educar a la comunidad en reconocer de manera oportuna los síntomas de un ACV para acudir rápidamente a un centro médico.\\

\par El trabajo “A novel end-to-end classifier using domain transferred deep convolutional neural networks for biomedical images” del año 2017 \cite{Pang2017}, aplica el método de la Redes Neuronales Convolucionales del DL. El objetivo es la clasificación de imágenes biomédicas y la identificación de enfermedades a partir de ellas. En el estudio, se propuso un clasificador de extremo a extremo altamente confiable y preciso para todo tipo de imágenes biomédicas a través del DL y el aprendizaje por transferencia. Como conclusión, el clasificador de extremo a extremo automatizado basado en un modelo con Redes Neuronales Convolucionales es altamente confiable y preciso que ha sido confirmado por varios conjuntos de datos de imágenes biomédicas públicas.\\

\par En el trabajo “Desafios bioéticos do uso da inteligência artificial em hospitais” del año 2022 \cite{Bioetica2022}, plantea un análisis de los desafíos de la IA en los hospitales. El objetivo es la identificación de desafíos en el desarrollo de sistemas dotados de IA (fase prehospitalaria) y en la implementación y formación de equipos de salud (fase hospitalaria). Como conclusión final, La literatura presentó numerosas posibilidades para el uso de la IA en el área de la salud, destacando su uso en el soporte hospitalario y sopesando las ventajas y desafíos.\\

\par La investigación “An improved Deep learning model for traffic crash prediction” del año 2018 \cite{shao2018improved}, utilizo DL con un modelo binomial negativo multivariable (MVNB). En este estudio, se propone un modelo de DL mejorado para explorar las complejas interacciones entre las carreteras, el tráfico, los elementos ambientales y los accidentes de tráfico. Como conclusión, el modelo propuesto que incluye la capa de regresión MVNB en el módulo de ajuste fino supervisado puede explicar mejor los patrones de distribución diferencial en los accidentes de tráfico según la gravedad de las lesiones y proporciona mejores predicciones de accidentes de tráfico.\\

\par Los trabajos más citados y precisos en el área de la salud y otros campos dentro de la IA se atribuyen al DL, la utilidad y precisión de este modelo es en gran medida funcional y hace confirmar hallazgos con características de vital importancia. Dentro de las técnicas clásicas la literatura nos hace referencia a las Artificial Neural Networks, Logistic Regression, Decision Tree, Random Forest y Naïve Bayes. En salud las escalas que miden en que estado se encuentra el paciente juegan un rol importante para una atención primaria de rápida atención, es por eso que la literatura señala a la escala NIHSS como una de las más importante para seguir avanzando en el algoritmo de atención en caso de un ACV.\\


\doublespacing
\chapter{DISCUSIONES, CONCLUSIONES Y RECOMENDACIONES}
\spacing{1.5}

\par En la investigación se realizó un estudio sobre exámenes o resultados de pacientes post Accidente Cerebrovascular del hospital Herminda Martin de Chillán por medio de técnicas de ML. La recopilación de datos de los pacientes fue una colaboración entre la investigadora e Informática Dra. Carolina Figueroa con el médico e investigador Dr. Carlos Escudero.\\

\par Por medio del trabajo, se concluye que existe relación entre el estado de alta del paciente y los datos que están en la base de datos, lo cual ha sido confirmado por los modelos de ML. Asimismo, los resultados son menores en relación con investigaciones anteriores, las cuales fueron expuestas en los trabajos relacionados, bordeando el promedio de predicción en un 90\% para las otras investigaciones y el mejor modelo elaborado en este trabajo un 83\%. Este resultado prueba la validez de la hipótesis de la investigación, asimismo, permite determinar la técnica de ML más precisa, con el fin de clasificar según el tipo de secuela a los pacientes que han sufrido ACV Isquémico del Hospital Herminda Martín. En otras palabras, la clasificación para conocer si un paciente tendrá un pronóstico de secuelas que le permitan tener un buen diario vivir o no, a través del modelo más confiable que es en este caso Decision Tree con un 83\% de probabilidades de precisión.\\

\par Para investigar las técnicas clásicas de ML se lograron estudiar varios modelos que fueron expuesto en el Marco Teórico y aplicaciones en los Trabajos Relacionados. Las técnicas con algoritmos supervisado existían más trabajos que estaban aplicados en la salud. Debe señalarse que el DL era la mejor opción, pero no se podía aplicar por escases de datos que se poseía.\\

\par En resumidas cuentas, las técnicas eran muy variadas y su utilización estaba optimizada en muchos casos con hiperparametros, aunque los modelos que más aparecieron en la literatura fueron Logistic Regression, Naive Bayes, Decision Tree y Random Forest, siendo que estos algoritmos se desenvuelven bien en problemáticas de salud y algunos actúan mejor con variables binarias para una mejor predicción.\\

\par Se estableció que el mecanismo para que el modelo llegará a poder predecir debía ser sacado de la literatura e investigaciones en salud de ML de código libre, en este caso particular se guio la investigación por el libro “Machine Leraning in Action” en su metodología y varios procesos de clasificaciones de datos. \\

\par Se determino que las variables de Hipertensión y Diabetes eran significativas para la investigación descrito en el capitulo 4 en la Dispersión de HTA y Diabetes respecto a la escala NISS de alta. Se demostró que las variables creadas en el proceso tenían una similitud parecida en importancia y que las que ya existían lo poseían también. Dependiendo el modelo hay variables que poseían mayor importancia para la predicción, como es el ejemplo de Decision Tree que contaba con la importancia más alta en la variable creada y procesada de “NIHSS\_alta\_cat“ con un 41\%, la seguía “NIHSS alta ACV” con un 21\%. En cambio, Random Forest su variable más importante era “TRIGLICERIDOS” con un 14\% y lo seguía la “EDAD” con un 9\%.\\

\par En cuanto a la comparativa de las técnicas de ML, en primera instancia con qué estado tenían un mejor desempeño, siendo que en el estado de Pronóstico Favorable el ganador fue “Decision Tree” con 249 puntos de 300 totales, destacamos que los resultados estuvieron parejos en este estado y en el estado de Pronóstico poco Favorable el ganador fue “Decision Tree” arrasando ante sus competidores con los mismo 249 puntos sobre 300.  En segunda instancia, la comparación final solo se tomo en cuanta la métrica de la Precisión Acumulada, también llevándose el primer lugar “Decision Tree” con un 83\% sobre su competidor más cercano Random Forest con un 62\% de predicción de los datos.\\

\par En el Procesamiento de los datos se utilizaron técnicas de clasificación de Missing Data, Clasificaciones propias de las variables, Label Encoding, entre otras, que proponen disminuir la perdida de datos, darle etiquetas a los valores y las etiquetas asignarles valores numéricos, respectivamente.
El One-Hot Encoding es una técnica que en la mayoría de las modelos de ML de clasificación es usa como paso anterior al entrenamiento de los datos en el modelo, eliminando la variable codificada entera y se agrega una nueva variable binaria para cada valor entero único, ayudando a que las variables pasen a un estado binario.
Cabe concluir, que este trabajo se enfocó en representar a los pacientes para tener una esperanza favorable o menos favorable en la escala NIHSS, contemplando el daño neurológico en los pacientes post ACV Isquémico, dando como resultado general que la mayoría de los pacientes post ACV Isquémico tendrán un pronóstico favorable cuando están de alta, enfatizamos que el alta no indica que el paciente se encuentre bien de salud, ya que el daño neurológico corresponde a perdida cognoscitiva o motor. Señalamos que aunque la muestra fue pequeña, se pueden obtener resultados en las posibles variables de importancia para nuevos trabajos investigativos en el campo de la medicina y ML.\\

\par Finalmente, los aportes de la IA a la medicina pueden ser un cambio que favorecerá para la respuesta más instantánea de la toma de decisiones de los médicos, pero debe haber una cooperación entre ambos campos de manera abierta, con inversiones para la salud, contando con equipos de trabajos y bases de datos gigantescas pero bien clasificadas como se menciona en los trabajos relacionados, esto llevará a poder dar un mejor pronóstico teniendo en cuenta que los equipos multidisciplinarios realizan grandes avances e involucran más aspectos que son necesarios para evaluar el sistema completo que lo compone a la persona. 
